%!TEX root = 03-tehnici de demonstratie.tex


\begin{section}{Formalizare}
\begin{frame}{Cum demonstrăm formal o proprietate?}
\begin{block}{Evaluarea expresiilor aritmetice}
Orice expresie aritmetică în IMP se evaluează folosind mașina SMC la un rezultat întreg
\begin{itemize}
\item depinzând (doar) de starea memoriei
\item Într-un număr finit de pași
\item Fără a modifica restul configurației
\end{itemize}
\end{block}

\onslide<2>
\only<beamer|handout>{
\begin{theorem}
Pentru orice expresie aritmetică $E$ în IMP și pentru  orice configurație $\c{c,r,s}$ a mașinii SMC în care $s$ e definită pentru toate locațiile lui $E$, există un număr întreg $n$ astfel încât 
$$\c{E\cdot c,  r, s} \longrightarrow^\ast \c{c, n\cdot r, s}$$
\end{theorem}}
\end{frame}

\begin{frame}{Inducție?}
\begin{theorem}
Pentru orice expresie aritmetică $E$ în IMP și pentru  orice configurație $\c{c,r,s}$ a mașinii SMC în care $s$ e definită pentru toate locațiile lui $E$, există un număr întreg $n$ astfel încât 
$$\c{E\cdot c,  r, s} \longrightarrow^\ast \c{c, n\cdot r, s}$$
\end{theorem}

\begin{block}{Demonstrație (Ciornă)}
Inducție după "structura" lui $E$
\begin{itemize}
\item Ce înseamnă structura lui $E$?
\item Pot face inducție după ea?
\end{itemize}
\end{block}
\end{frame}

\end{section}

\begin{section}{Sintaxă abstractă}
\begin{frame}{Arbori de sintaxă}
   \begin{block}{Parsare}
      Program \hfill $\xrightarrow{\mbox{Parsare}}$ \hfill Arbore de derivare\\
     $3 + ! x$ \hfill 
\begin{minipage}{.52\columnwidth}
\graph{
  \Vertex[L={$E$},style={color=blue!75!black}]{E}
  \SOWE[L={$E$},style={color=blue!75!black}](E){E1} 
  \SOEA[L={$E$},style={color=blue!75!black}](E){E2} 
  \SO[L={$\it iop$},style={color=blue!75!black}](E){Op} 
  \SO[L={$n$},style={color=blue!75!black}](E1){n} 
  \SO[L={$+$}](Op){P}
  \SO[L={$!$}](E2){B} 
  \SOEA[L={$l$},style={color=blue!75!black}](E2){L} 
  \SO[L={$3$}](n){N3}
   \SO[L={$x$}](L){x}
  \Edge(E)(E1)
   \Edge[label={$E$ ::= $E$ $\it iop$ $E$},labelstyle={right=2em,midway,inner sep=0pt,minimum size=0pt}](E)(E2)
  \Edge(E)(Op)
  \Edge[label={$E$ ::= $n$},labelstyle={left=1em,midway,inner sep=0pt,minimum size=0pt}](E1)(n)
   \Edge(n)(N3)
  \Edge(Op)(P)
   \Edge(E2)(B)
   \Edge[label={$E$ ::= ``!'' $l$},labelstyle={right=2em,midway,inner sep=0pt,minimum size=0pt}](E2)(L)
   \Edge(L)(x)
}\end{minipage}
   \end{block}
\onslide<2>
\begin{block}{Arbore abstract de sintaxă (AST)}
\vspace*{-2ex}\begin{minipage}{.59\columnwidth}
\begin{itemize}
\item Pune în evidența structura
\item Ignoră detaliile „gramaticale“
\item „iop“, „int“, „loc“ se numesc \structure{constructori}
\end{itemize}
\end{minipage}\hfill
\begin{minipage}{.2\columnwidth}
\vspace*{-1ex}
\graph{
  \Vertex[L={iop},style={color=blue!75!black}]{E}
  \SOWE[L={int},style={color=blue!75!black}](E){E1} 
  \SOEA[L={loc},style={color=blue!75!black}](E){E2} 
  \SO[L={$+$}](E){Op} 
  \SO[L={$3$}](E1){n} 
  \SO[L={$x$}](E2){B} 
  \Edge(E)(E1)
   \Edge(E)(E2)
  \Edge(E)(Op)
  \Edge(E1)(n)
   \Edge(E2)(B)
}\end{minipage}
\end{block}
\end{frame}

\begin{subsection}{Tip abstract de date}

\begin{frame}[fragile]{\only<beamer|handout>{Tip abstract de date}}{Reprezentăm programele ca termeni algebrici}
\begin{block}{Signatură algebrică (în Maude)}
\begin{asciik}
  sorts IExp BExp Cmd . 
  op int : Int -> IExp .                   --- integer injection
  op iop : Iop IExp IExp -> IExp .	        --- integer operation 
  op loc : Loc -> IExp .                   --- location !
  op bool : Bool -> BExp .                 --- boolean injection
  op bop : Bop IExp IExp -> BExp .         --- integer operation 
  op skip : -> Cmd .                       --- skip
  op seq : Cmd Cmd -> Cmd .                --- seq composition ;
  op asgn : Loc IExp -> Cmd .              --- assignment :=
  op if : BExp Cmd Cmd -> Cmd .            --- conditional
  op while : BExp Cmd -> Cmd .             --- while loop
\end{asciik}
\end{block}
\end{frame}

\begin{frame}[fragile]{Reprezentarea AST ca termen}
\begin{block}{Program\hspace{8em} Termen AST}
\begin{minipage}{.3\columnwidth}
\begin{asciic}
 x := 10 ;
(sum := 0 ;
 while (0 <= !x)  do 
   (sum := !sum + !x ;
    x := !x - 1
   )
 )
\end{asciic}
\end{minipage}\hfill
\begin{minipage}{.64\columnwidth}
\footnotesize
\begin{verbatim}
 seq(asgn(x,int(10)),
 seq(asgn(sum,int(0)),
     while(bop(<=, int(0), loc(x)),
       seq(asgn(sum, iop(+, loc(sum), loc(x))),
           asgn(x, iop(-, loc(x), int(1)))
       ))
 ))
\end{verbatim}
\end{minipage}
\end{block}
\begin{itemize}
\item Pentru a ușura notația lucrăm cu reprezentarea sintactică (stânga)
\item Folosim paranteze pentru dezambiguizare
\item Asociem (mental) fiecărei fraze program termenul AST corespunzător
\item Identificăm termenul AST cu arborele asociat
\end{itemize}
\end{frame}
\end{subsection}
\end{section}

\begin{section}{Inducție matematică}
\begin{frame}{Inducție slabă/incompletă}
Pentru a demonstra că $P(n)$ este adevărată pentru orice număr natural $n$:
\onslide<2>
\begin{description}
\item[Caz de bază] Demonstrăm că $P(0)$ e adevărată
\item[Pas de inducție] Pentru orice număr natural $n$, demonstrăm că \\dacă $P(n)$ e adevărată, atunci și $P(n+1)$ este adevărată 
\end{description}
\end{frame}

\begin{frame}{Inducție tare/completă}
Pentru a demonstra că $P(n)$ este adevărată pentru orice număr natural $n$:
\onslide<2>
\begin{description}
\item[Pas de inducție] Pentru orice număr natural $n$, demonstrăm că \\dacă $P(i)$ e adevărată pentru orice număr natural $i < n$, \\atunci și $P(n)$ este adevărată 
\end{description}
\end{frame}


\begin{frame}{Inducție structurală}
Pentru a demonstra că $P(t)$ este adevărată pentru orice AST (termen) $t$ dat de o gramatică (tip abstract de date):
\begin{description}
\item[Pas de inducție] Pentru orice AST (termen) valid $t = c(t_1,\ldots,t_n)$, demonstrăm că dacă $P(t_i)$ e adevărată pentru orice $i=\overline{1,n}$ pentru care $t_i$ de acelasi tip ca si $t$, atunci și P(t) e adevărată
\end{description}

\begin{block}{Observație}
Demonstrațiile prin inducție structurală se reduc la analize de caz
\end{block}
\end{frame}

\begin{subsection}{Aplicații}
\begin{frame}<handout:0|article:0>{Inducție structurală}{Exemplu}
\begin{theorem}
Pentru orice expresie aritmetică $e$ în IMP și pentru  orice configurație $\c{c,r,s}$ a mașinii SMC în care $s$ e definită pentru toate locațiile lui $e$, există un număr întreg $n$ astfel încât 
$$\c{e\cdot c,  r, s} \longrightarrow^\ast \c{c, n\cdot r, s}$$
\end{theorem}
\begin{block}{Reguli SMC pentru expresii aritmetice $E$ ::= $n \Smid {\mathrel{"!"} l} \Smid E \mathrel{iop} E$}
$\begin{array}{l@{\extracolsep{1em}}r@{{}\longrightarrow{}}l}
\mbox{IntCt}&
\c{n\cdot c,r,s} & \c{c, n \cdot r, s}
\\[1ex]
\mbox{Loc}&
\c{\terminal{!} l\cdot c,r,s} & \c{c, n \cdot r, s}  \Sdaca s(l) = n
\\[1ex]
\mbox{IntOpC}&
\c{(E_1 \iop E_2) \cdot c, r, s} & \c{E_1 \cdot E_2 \cdot \iop \cdot c, r, s}
\\[1ex]
\mbox{IntOp}&
\c{\iop \cdot c, n_2\cdot n_1 \cdot r, s} & \c{c, n \cdot r, s} \Sdaca n_1 \iop n_2 = n
\end{array}$
\end{block}
\end{frame}

\begin{frame}{\only<beamer|handout>{Inducție structurală}}{Exemplu}
  \setcounter{theorem}{0}
\begin{theorem}
Pentru orice expresie aritmetică $e$ în IMP și pentru  orice configurație $\c{c,r,s}$ a mașinii SMC în care $s$ e definită pentru toate locațiile lui $e$, există un număr întreg $n$ astfel încât 
$$\c{e\cdot c,  r, s} \longrightarrow^\ast \c{c, n\cdot r, s}$$
\end{theorem}

\begin{block}{Demonstrație (%
Inducție structurală asupra  lui $e$)\hfill  $E$ ::= $n \Smid {\mathrel{"!"} l} \Smid E \mathrel{iop} E$}
\onslide<2>
\begin{enumerate}
\item \structure{$e = n$}, unde $n$ întreg, atunci aplicăm axioma (IntCt) și avem 
$\c{n\cdot c,  r, s} \xrightarrow{IntCt} \c{c, n\cdot r, s}$
\item \structure{$e = \mathop{!} l$}, unde $l$ locație, atunci aplicăm axioma (Loc) și avem 
$\c{! l\cdot c,  r, s} \xrightarrow{Loc} \c{c, n\cdot r, s}$, unde $n=s(l)$ există  ($s$ definit pentru $l$)
\end{enumerate}
\end{block}
\end{frame}

\begin{frame}<handout:0|article:0>{Inducție structurală}{Exemplu}
\begin{theorem}
Pentru orice expresie aritmetică $e$ în IMP și pentru  orice configurație $\c{c,r,s}$ a mașinii SMC în care $s$ e definită pentru toate locațiile lui $e$, există un număr întreg $n$ astfel încât 
$$\c{e\cdot c,  r, s} \longrightarrow^\ast \c{c, n\cdot r, s}$$
\end{theorem}
\begin{block}{Reguli SMC pentru expresii  aritmetice $E$ ::= $n \Smid {\mathrel{"!"} l} \Smid E \mathrel{iop} E$}
$\begin{array}{l@{\extracolsep{1em}}r@{{}\longrightarrow{}}l}
\mbox{IntCt}&
\c{n\cdot c,r,s} & \c{c, n \cdot r, s}
\\[1ex]
\mbox{Loc}&
\c{\terminal{!} l\cdot c,r,s} & \c{c, n \cdot r, s}  \Sdaca s(l) = n
\\[1ex]
\mbox{IntOpC}&
\c{(E_1 \iop E_2) \cdot c, r, s} & \c{E_1 \cdot E_2 \cdot \iop \cdot c, r, s}
\\[1ex]
\mbox{IntOp}&
\c{\iop \cdot c, n_2\cdot n_1 \cdot r, s} & \c{c, n \cdot r, s} \Sdaca n_1 \iop n_2 = n
\end{array}$
\end{block}
\end{frame}



\begin{frame}{\only<beamer|handout>{Inducție structurală}}{\only<beamer|handout>{Exemplu}}
\only<beamer|handout>{
\begin{theorem}
Pentru orice expresie aritmetică $e$ în IMP și pentru  orice configurație $\c{c,r,s}$ a mașinii SMC în care $s$ e definită pentru toate locațiile lui $e$, există un număr întreg $n$ astfel încât 
$$\c{e\cdot c,  r, s} \longrightarrow^\ast \c{c, n\cdot r, s}$$
\end{theorem}}

\begin{block}{\only<beamer|handout>{Demonstrație (%
Inducție structurală asupra  lui $e$)\hfill  $E$ ::= $n \Smid {\mathrel{"!"} l} \Smid E \mathrel{iop} E$}}
\begin{enumerate}
  \setcounter{enumi}{2}
  \item 
\structure{$e = e_1 \iop e_2$}, unde $e_1, e_2$ expresii
\onslide<2>
\begin{itemize}
\item $\c{(e_1 \iop e_2) \cdot c, r, s}  \xrightarrow{IntOpC} \c{e_1 \cdot e_2 \cdot \iop \cdot c, r, s}$
\item $\c{e_1 \cdot e_2 \cdot \iop \cdot c, r, s} \longrightarrow^\ast \c{e_2 \cdot \iop \cdot c, n_1 \cdot r, s}$ \hfill (ip. inducție)
\item $\c{e_2 \cdot \iop \cdot c, n_1 \cdot r, s}  \longrightarrow^\ast \c{\iop \cdot c, n_2 \cdot n_1 \cdot r, s} $ \hfill (ip. inducție)
\item $\c{\iop \cdot c, n_2 \cdot n_1 \cdot r, s}\xrightarrow{IntOp} \c{c, n \cdot r, s}$, unde $n_1 \iop n_2 = n$
\end{itemize}
\end{enumerate}
\end{block}
\end{frame}

\begin{frame}{Ce putem/vrem să mai demonstrăm}
\begin{block}{Evaluarea expresiilor booleene}
Orice expresie booleană în IMP se evaluează folosind mașina SMC la un rezultat boolean
\begin{itemize}
\item depinzând (doar) de starea memoriei
\item Într-un număr finit de pași
\item Fără a modifica restul configurației
\end{itemize}
\end{block}
\begin{block}{Evaluarea instrucțiunilor simple}
Orice program IMP care nu conține while se termină
\begin{itemize}
\item Modificând doar conținutul memoriei
\end{itemize}
\end{block}

\end{frame}

\end{subsection}
\end{section}
