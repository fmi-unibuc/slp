%!TEX root = 04-semantica evaluarii.tex

\begin{frame}{Mașini abstracte}{Avantaje și dezavantaje}
\begin{block}{Puncte forte}
\begin{itemize}
\item Ușor (și eficient) de implementat
\item (Relativ) ușor de manipulat stivele de control și date
\item Foarte flexibilă --- limbajul stivelor fiind la alegerea celui ce o definește
\end{itemize}
\end{block}
\vfill
\begin{block}{Puncte slabe}
\begin{itemize}
\item Multe tranziții au doar scop sintactic, ne-computațional
\item Datorită îmbogățirii limbajului, multe configurații sunt irelevante 
\item Prea concretă si flexibilă, ceea ce îngreunează folosirea ei pentru raționamente formale
\end{itemize}
\end{block}
\end{frame}

\begin{frame}{Semantică operațională \structure{structurată (SOS)}}
{Concept (Plotkin, 1981) }

\begin{itemize}
\vitem Definiția sistemului de tranziție trebuie să reflecte structura limbajului
\item Tranzițiile pentru elemente compuse sa fie definite recursiv 
\begin{itemize}
\item în funcție de tranzițiile pe componente
\end{itemize}
\item Raționamente formale asupra sistemului prin inducție


\vitem Execuția ca demonstrație 
\item Controlul execuției este dat (intuitiv) de „firul deomnstrației“

\vitem Claritate prin abstractizare
\item Stivele de control și rezultate nu mai sunt necesare
\end{itemize}
\end{frame}

\begin{section}{Definiții recursive}

\begin{frame}{\only<beamer>{Definiții recursive}}{Exemple}
  \begin{block}{Numerele naturale (Peano)}
      \begin{itemize}
      \item $0 \in \mathbb{N}$
       \item Dacă $n \in \mathbb{N}$ atunci și $s(n) \in \mathbb{N}$
      \end{itemize}
  \end{block}
  \begin{block}{Limbaj descris de o gramatică \hfill 
\begin{minipage}{.35\columnwidth} \renewcommand{\syntaxKeyword}{}\syntax[\AExp]{\Int\Smid\terminal{!}\Id \Smid \AExp\iop \AExp}{}\end{minipage}}
     \begin{itemize}
	 \item Orice număr întreg $n$ este expresie
         \item Pentru orice locație $l$, $\mathrel{!} l$ este expresie
         \item Pentru orice simbol de operație întreagă $o$\\
              Dacă $e_1$ și $e_2$ sunt expresii, atunci și $e_1 \mathrel{o} e_2$ este expresie
      \end{itemize}
  \end{block}
  \begin{itemize} 
        \item Arbori de parsare/arbori AST
        \item Termeni AST descriși de un tip abstract de date
       \item Termeni descriși de o signatură algebrică
\end{itemize}
\end{frame}

\begin{frame}{Reguli de deducție}
   \begin{block}{Definiție}
        O \structure{regulă de deducție} $r$ peste o mulțime T este o pereche $(H,c)$ unde 
         \begin{itemize}
             \item $H \subseteq T$ este mulțimea de \structure{ipoteze} ale regulii
             \item $c\in T$ este \structure{concluzia} regulii
         \end{itemize}         
        Dacă mulțimea $H$ a ipotezelor e vidă, $r$ se numește \structure{axiomă}.

        \structure{Prezentare arborescentă:}  $\frac{h_1\ h_2\ \cdots\ h_n}{c}$ pentru reguli și $\frac{\done}{c}$ pentru axiome.
   \end{block}
   \begin{block}{Exemplu (Peano):}
      Fie $T = \{0,s,(,)\}^\star$ limbajul tuturor cuvintelor peste alfabetul $\{0, s,(,)\}$
      \[{\cal R} = \{ (\emptyset, 0) \} \cup \{ (\{t\}, s ( t ) ) \mid t \in T\}\]
    \end{block}
\end{frame}

\begin{frame}{Scheme de reguli de deducție}
\begin{itemize}
 \item
   Reguli cu variabile și constrângeri asupra lor: 
$\displaystyle\reg{c}{h_1 \cdots h_n}{\mbox{condiții}}$
\item O schemă de reguli reprezintă mulțimea tuturor instanțelor obținute prin substituirea variabilelor astfel încât eventualele condiții să fie satisfăcute.
\end{itemize}
   \begin{block}{Exemplu (Peano):}
 $\displaystyle\hfill \reg{0}{}{} \hfill \reg{s(t)}{t}{}\hfill\ $
\end{block}
   \begin{block}{Exemplu (expresii aritmetice):} 
\begin{itemize}
\item[]
              \structure{Axiome:} \hfill $\displaystyle\reg{n}{}{n \mbox{ intreg}}$  \hfill $\displaystyle\reg{\mathrel{!} l}{}{l\mbox{ locație}}$\hfill\ 
  

\item[]              \structure{Regulă:} \hfill $\displaystyle\reg{e_1\mathrel{o} e_2}{e_1\si  e_2}{o \mbox{ operație întreagă}}$\hfill\;
\end{itemize}
       \end{block}
\end{frame}

\begin{frame}{Demonstrații deductive}{Prezentare ca arbori}
   \begin{block}{Definiție}
       Un \structure{arbore de demonstrație} folosind regullile $\cal R$ peste mulțimea $T$ satisface:
     \begin{itemize} 
         \item Orice nod este etichetat cu un element din $T$
        \item Daca multimea etichetelor copiilor unui nod este $H$ și eticheta nodului este $c$, atunci $(H,c) \in \cal R$
         \item Înălțimea arborelui este finită
      \end{itemize}
    Un arbore de demonstrație e o demonstrație pentru eticheta rădăcinii lui.
    \end{block}

 \begin{block}{Exemplu (expresii aritmetice)}
     \[\reg{(3 + ! x) * 5 }{
            {\displaystyle\reg{3 + ! x}{\displaystyle\reg{3}{}{}\si \reg{ ! x}{}{}}{}\si \reg{5}{}{}}{}}{}\]
  \end{block}
\end{frame}

\begin{frame}{Demonstrații deductive}{Prezentare liniară}
   \begin{block}{Definiție}
       O \structure{demonstrație} a lui $c$ folosind mulțimea de reguli $\cal R$  este o secvență
de reguli $(H_1, c_1), \ldots (H_n, c_n)$ satisfacând următoarele    
     \begin{itemize}
          \item $(H_i, c_i) \in \cal R$ pentru orice $1\leq i \leq n$
          \item $c_n = c$
          \item Orice ipoteză $h \in H_i$ a fost deja demonstrată, i.e., există $c_j = h$ cu $j<i$ 
     \end{itemize}
    \end{block}

  \begin{block}{Exemplu}
      $(\emptyset, 3)$,
      $(\emptyset, 5)$,
      $(\emptyset, ! x)$,
      $(\{3, !x \}, 3 + ! x)$,
     $(\{3 + ! x, 5\}, (3 + ! x) * 5)$.
  \end{block}
\end{frame}

\begin{frame}{Definiții recursive}{Definiție formală}
  \begin{block}{Definiție}
     Mulțimea \structure{definită recursiv} de regulile {\cal R} este submulțimea elementelor $c$ ale lui $T$ demonstrabile folosind $\cal R$.
  \end{block}

\begin{itemize}
  \item Denumire alternativă: mulțimea \structure{teoremelor} lui ${\cal R}$
  \item Cea mai mică mulțime A „închisă“ la regulile din $\cal R$, i.e.,  satisfăcând:
  \begin{itemize}
      \item pentru orice regulă $(H,c) \in {\cal R}$, dacă $H\subseteq A$ atunci $c\in A$.
  \end{itemize}
\end{itemize}

\end{frame}


\end{section}

\begin{section}{Semantică naturală}
\begin{frame}{Semantica naturală / a evaluării}{Expresii aritmetice}
Definim recursiv graficul relației de evaluare ca mulțime de perechi $(e, s, n)$
\begin{itemize}
  \item $e$ --- expresie, $s$ --- stare a memorie, $n$ --- valoare întreagă
  \item Notăm $\S{\c{e,s}}{n}$ pentru a sugera ideea de funcție de evaluare
\end{itemize}

\vfill\begin{block}{Axiome și reguli}
\begin{itemize}
\item[] $\reg[int]{\S{\c{n,s}}{n}}{}{n \mbox{ întreg}}$

\item[]$\reg[loc]{\S{\c{{\mathrel{!} l},s}}{n}}{}{l \mbox{ locație, cu $l\in Dom(s)$ și $s(l)=n$}}$

\item[]$\reg[iop]{\S{\c{e_1\mathrel{o} e_2,s}}{n}}{\S{\c{e_1,s}}{n_1}\si \S{\c{e_2,s}}{n_2}}{o \mbox{ operație întreagă și $n = n_1 \mathrel{o} n_2$}}$
\end{itemize}
\end{block}
\end{frame}

\begin{frame}{Inducție deductivă}
Pentru a demonstra că $P(t)$ este adevărată pentru orice element al unei mulțimi $A$ definite recursiv de regulile $\cal R$:
\begin{itemize}
  \item Considerăm mulțimea elementelor $t$ din $A$ pentru care $P(t)$ e adevărată
 \item Arătăm că este închisă la regulile din $\cal R$
\end{itemize}

\begin{block}{Concret}
Pentru fiecare regulă $(H,c)$, demonstrăm că
\begin{itemize} 
\item Dacă $P(h)$ e adevărată pentru orice ipoteză $h\in H$,
\item atunci și $P(c)$ e adevărată.
\end{itemize}	
\end{block}
\end{frame}

\begin{frame}{Inducție deductivă}{Exemplu}
\begin{theorem}[Corespondență între semantica naturală și mașina SMC]
Dacă $\S{\c{e,s}}{n}$ atunci $\c{e\cdot c, r, s} \longrightarrow^\ast \c{c, n\cdot r, s}$.
\end{theorem}
\begin{block}{Demonstrație (inducție deductivă asupra relației $\S{}{}$)}
  \begin{description} 
     \item[$\reg{\S{\c{n,s}}{n}}{}{n \mbox{ întreg}}$:] $\c{n \cdot c, r, s} \xrightarrow{\mbox{IntCt}} \c{c, n\cdot r, s}$ 
    \item[$\reg{\S{\c{{\mathrel{!} l},s}}{n}}{}{l \mbox{ locație, cu $l\in Dom(s)$ și $s(l)=n$}}$ :] 
$\c{{\mathrel{!} l} \cdot c, r, s} \xrightarrow{\mbox{Loc}} \c{c, n\cdot r, s}$ 
    \item[$\reg{\S{\c{e_1\mathrel{o} e_2,s}}{n}}{\S{\c{e_1,s}}{n_1}\si \S{\c{e_2,s}}{n_2}}{o \mbox{ operație întreagă și $n = n_1 \mathrel{o} n_2$}}$ :]

     \item[]\vspace{-1ex}$\begin{array}{l}
  \c{e_1\mathrel{o} e_2 \cdot c, r, s} \xrightarrow{\mbox{IntOpC}} \c{e_1 \cdot e_2 \cdot o \cdot c, r, s} \xrightarrow{\mbox{ip. ind}}
\\
  \c{e_2 \cdot o \cdot c, n_1 \cdot r, s}  \xrightarrow{\mbox{ip. ind}}   \c{o \cdot c, n_2 \cdot n_1 \cdot r, s}  \xrightarrow{\mbox{IntOp}}
\\
   \c{c, n \cdot r, s}
\end{array} $
  \end{description}
\end{block}
\end{frame}

\end{section}

\begin{section}{O semantica naturală pentru limbajului IMP}
\begin{frame}{Evaluarea fragmentelor de programe}
\begin{itemize}
\item Expresiile (aritmetice sau booleene) se evaluează la valori
\item Comenzile modifică starea
\vitem Considerăm „valoare“ a execuției unui (fragment de) program o pereche $\c{v,s}$, unde
\begin{itemize}
\item $s$ este starea obținută după execuție
\item $v$ este un întreg (pentru expresii aritmetice), o valoare booleană (pentru comparații) sau {\tt skip} (pentru comenzi)
\end{itemize}
\vitem Definim recursiv relația de evaluare $\S{\c{p,s}}{\c{v,s'}}$
\end{itemize}
\end{frame}

\begin{frame}{Evaluarea expresiilor}

\begin{itemize}
\item[] $\reg[ct]{\S{\c{v,s}}{\c{v,s}}}{}{v \mbox{ întreg sau boolean}}$

\vitem[]
$\reg[loc]{\S{\c{{\mathrel{!} l},s}}{\c{n,s}}}{}{l \mbox{ locație, cu $l\in Dom(s)$ și $s(l)=n$}}$

\vitem[]
$\reg[op]{\S{\c{e_1\mathrel{o} e_2,s}}{\c{v,s_2}}}{\S{\c{e_1,s}}{\c{n_1, s_1}}\si \S{\c{e_2,s_1}}{\c{n_2, s_2}}}{v = n_1 \mathrel{o} n_2}$

\end{itemize}

\end{frame}

\begin{frame}{Evaluarea comenzilor nerepetitive}

\begin{itemize}
\item[] $\reg[skip]{\S{\c{\Sskip,s}}{\c{\Sskip,s}}}{}{}$

\vitem[]
$\reg[asgn]{\S{\c{{l\mathrel{:=} e},s}}{\c{\Sskip,s''}}}{\S{\c{e,s}}{\c{n,s'}}}{s'' = s'[ l \mapsto n]}$

\vitem[]
$\reg[seq]{\S{\c{c_1\mathrel{;} c_2,s}}{\c{\Sskip,s_2}}}{\S{\c{c_1,s}}{\c{\Sskip, s_1}}\si \S{\c{c_2,s_1}}{\c{\Sskip, s_2}}}{}$

\vitem[]
$\reg[if-true]{\S{\c{\Sif b \Sthen c_1 \Selse c_2,s}}{\c{\Sskip,s_1}}}{\S{\c{b,s}}{\c{\Strue, s'}}\si \S{\c{c_1,s'}}{\c{\Sskip, s_1}}}{}$

\vitem[]
$\reg[if-false]{\S{\c{\Sif b \Sthen c_1 \Selse c_2,s}}{\c{\Sskip,s_2}}}{\S{\c{b,s}}{\c{\Sfalse, s'}}\si \S{\c{c_2,s'}}{\c{\Sskip, s_2}}}{}$


\end{itemize}
\end{frame}

\begin{frame}{Evaluarea lui $\Swhile$}
\ 

\medskip\vfill
$\reg{\S{\c{\Swhile b \Sdo c,s}}{\c{\Sskip,s'''}}}{\S{\c{b,s}}{\c{\Strue, s'}}\si \S{\c{c,s'}}{\c{\Sskip, s''}}\si \S{\c{\Swhile b \Sdo c,s''}}{\c{\Sskip,s'''}}}{}$

\hfill ({\sc while-true})

\vfill
$\reg[while-false]{\S{\c{\Swhile b \Sdo c,s}}{\c{\Sskip,s'}}}{\S{\c{b,s}}{\c{\Sfalse, s'}}}{}$

\vfill
\end{frame}

\begin{frame}{Ce putem/vrem să demonstrăm}
\begin{itemize}
\item Corespondența completă între semantica naturală și mașina SMC: dacă $\S{\c{St, s}}{\c{\Sskip, s'}}$ atunci \[\c{St\cdot c, r, s} \longrightarrow^\ast \c{c, r, s'}\]
\vitem Programul $c$ dat de $\Swhile {\Strue} \Sdo {\Sskip}$ nu poate fi evaluat.  Nu există stările $s, s'$ și valoarea $v$ astfel încât $\S{\c{c,s}}{\c{v,s'}}$. 
\end{itemize}
\end{frame}


\end{section}
