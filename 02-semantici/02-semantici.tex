\documentclass[xcolor=pdftex,romanian,colorlinks]{beamer}

\usepackage{../tslides}
\usepackage{comment}

\title[SLP---Tipologie]{Semantica limbajelor de programare}
\subtitle{Tipuri de Semantică}
\begin{document}
\begin{frame}
  \titlepage
\end{frame}

\begin{frame}{Feluri de a da semantica}
\begin{itemize}
  \item Limbaj de programare: sintaxă și semantică
  \vitem Feluri de semantică
  \begin{itemize}
  \item Limbaj natural --- descriere textuală a efectelor
  \vitem Operațională --- asocierea unei demonstrații a execuției
  \vitem Axiomatică --- Descrierea folosind logică a efectelor unei instrucțiuni
  \vitem Denotațională --- prin asocierea unui obiect matematic (denotație)

  \vitem Statică --- Asocierea unui sistem de tipuri care exclude programe eronate
  \end{itemize}
\end{itemize}
\end{frame}

\begin{subsection}{IMP}
\begin{frame}[fragile]{Limbajul IMP}
IMP este un limbaj \structure{IMP}erativ foarte simplu.
\begin{block}{Ce conține}
\begin{minipage}{.49\columnwidth}
\begin{itemize}
\vitem Expresii
\begin{itemize}
  \item Aritmetice%\hfill $x + 3$
  \item Booleene%\hfill $\Snot(x > 7)$
\end{itemize}
\vitem Blocuri de instrucțiuni
\begin{itemize}
  \item De atribuire%\hfill $x = 5;$
  \item Condiționale%\hfill $\Sif(x > 7)\; \{x =5; \} \Selse \{x = 0;\}$
  \item De ciclare  %\hfill$\Swhile (x > 7)\; \{x = x - 1;\}$
\end{itemize}
\end{itemize}
\end{minipage}
\begin{minipage}{.49\columnwidth}
\begin{asciic}
int x = 10;
int sum = 0;
while (0 <= x) {
  sum = sum + x;
  x = x + -1;
}
\end{asciic}
\end{minipage}
\end{block}
\begin{block}{Ce nu conține}
\begin{itemize}
\item Expresii cu efecte laterale
\item Proceduri și funcții
\item Schimbări abrupte de control
\end{itemize}
\end{block}
\end{frame}

\begin{comment}
 \begin{frame}{Sintaxă formală}{Backus Naur Form}
  \begin{itemize}
   \item Pentru gramatici (generative) independente de context
   \item Producții
       --- generează termeni prin expandare (rescriere)

\alert{\renewcommand{\syntaxKeyword}{}\syntax[\Stmt]{\color{black}\Sif\;\terminal{(}\BExp\terminal{)}\Block\Selse\Block}{}
	\syntaxCont[\Stmt]{\color{black}\Id \terminal{=}\AExp\terminal{;}}{}}
   \item Categorii sintactice (neterminale, încep cu majusculă)
   \begin{itemize}
    \item descriu tipurile de sintaxă
     \begin{itemize}
      \item Tipuri lexicale: %$\Int$ (întregi), $\Id$ (identificatori), $\Bool$ (Booleeni)
	  \item Tipuri construite:
     %$\AExp$ (expr. aritmetice), $\BExp$ (expr. Booleene), \\$\Stmt$ (instrucțiuni), $\Block$ (bloc de instrucțiuni), $\Pgm$ (program)
     \end{itemize}
   \end{itemize}
   \item Cuvinte cheie (terminale, încep cu literă mică sau simboluri)
   \begin{itemize}
    \item descriu elementele lexicale:
   $\Sif$, $\Selse$, $\Sint$,
   $\Swhile$,
   $\terminal{<=}$,
   $\terminal{\&\&}$, $\terminal{!}$,
   $\terminal{+}$, $\terminal{=}$, $\terminal{;}$, $\terminal{(}$, $\terminal{)}$, \ldots
   \end{itemize}
  \end{itemize}
 \end{frame}
\end{comment}
\begin{frame}{Sintaxă formală}{Sintaxa BNF a limbajului IMP}
\vspace{-5ex}\begin{syntaxBlock}{\AExp}
\alert{
\begin{itemize}
\item[]\renewcommand{\syntaxKeyword}{}
\syntax{\Int\Smid\Id}{}
\syntaxCont{\AExp\terminal{+}\AExp\Smid\AExp\terminal{*}\AExp\Smid\AExp\terminal{/}\AExp}{}
\item[]\renewcommand{\defSort}{\BExp}
\syntax{\Bool}{}
\syntaxCont{\AExp\terminal{<=}\AExp}{}
\syntaxCont{\Snot\BExp \Smid \BExp \terminal{\&\&} \BExp}{}
\item[]\renewcommand{\defSort}{\Block}
\syntax{\terminal{\{}\Stmt\terminal{\}} \Smid \terminal{\{}\terminal{\}}}{}
\syntaxCont{\Stmt\terminal{}\Stmt}{}
\syntaxCont{\Id\terminal{=}\AExp\terminal{;}}{}
\syntaxCont{\Sif\;\terminal{(}\BExp\terminal{)}\Block\Selse\Block}{}
\syntaxCont{\Swhile\;\terminal{(}\BExp\terminal{)}\Block}{}
\item[]\renewcommand{\defSort}{\Pgm}
\syntax{\Sint \Id \terminal{=} \Int \terminal{;} \Pgm\Smid \Stmt}{}
\end{itemize}
}
\end{syntaxBlock}
\end{frame}
\end{subsection}

\begin{frame}{Semantică în limbaj natural}

\begin{block}{Atribuirea: \alert{$x	= {\rm expr}$}}
  \begin{itemize}
    \item<2> Expresia este evaluată în starea curentă a programului
	\item<2> Variabilei i se atribuie valoarea calculată, înlocuind valoarea precedentă a acelei variabile.
  \end{itemize}
\end{block}

\begin{block}<2>{Avantaje și dezavantaje}
\begin{itemize}
\item[+] Ușor de prezentat
\item[\alert{-}] Potențial ambiguă
\item[\alert{-}] Imposibil de procesat automat
\end{itemize}
\end{block}
\end{frame}

\begin{frame}{Semantica denotațională}
\begin{block}{Atribuirea: \alert{$x	= {\rm expr}$}}
\begin{itemize}
\item Asociem expresiilor aritmetice funcții de la starea memoriei la valori:
\begin{itemize}
\item<2> Funcția constantă $[[1]](s) = 1$
\item<2> Funcția care selectează valoarea unui identificator $[[x]](s) = s(x)$
\item<2> „Morfismul de adunare” $[[e1 + e2]](s) = [[e1]](s) + [[e2]](s)$.
\end{itemize}
\vitem Asociem instrucțiunilor funcții de la starea memoriei la starea (următoare) a memoriei.
\begin{itemize}
  \item<2> $[[x = e]](s)(y) = \left\{\begin{array}{r@{\mbox{, dacă }}l} s(y) & y \neq x
  \\{} [[e]](s) & y = x\end{array}\right.$
\end{itemize}
\end{itemize}
\end{block}
\begin{block}<2>{Avantaje și dezavantaje}
\begin{itemize}
\item[+] Formală, matematică, foarte precisă
\item[+] Compozițională (morfisme și compuneri de funcții)
\item[\alert{-}] Greu de stăpânit (domeniile devin din ce în ce mai complexe)
\end{itemize}
\end{block}

\end{frame}

\begin{frame}[fragile]{Semantica denotațională a lui IMP}
\begin{minipage}{.46\columnwidth}
\begin{asciihs}
type Id = String
type State = Id -> Int
type DAExp = State -> Int
type DBExp = State -> Bool
type DStmt = State -> State

denotAExp :: AExp -> DAExp
denotAExp (Int n) s = n
denotAExp (e1 :+: e2) s
  = denotAExp e1 s
  + denotAExp e2 s

denotBExp :: BExp -> DBExp
denotBExp (Bool b) s = b
denotBExp (a1 :<=: a2) s
  = denotAExp a1 s
  <= denotAExp a2 s

\end{asciihs}
\end{minipage}
\begin{minipage}{.53\columnwidth}
\begin{asciihs}
denotStmt :: Stmt -> DStmt
denotStmt Skip = id
denotStmt (s1 ::: s2) =
  denotStmt s2 . denotStmt s1
denotStmt (If c t e) s
 |denotBExp c s = denotStmt t s
 |otherwise     = denotStmt e s
denotStmt (While c b) =
  fix (\w s ->
    if denotBExp b s
      then w (denotStmt c s)
      else s)
\end{asciihs}
\end{minipage}
\end{frame}

\begin{frame}{Semantica Axiomatică}{Logica Floyd-Hoare}
  Definește o relație ternară de forma
  $\{Pre\}S\{Post\}$, unde:
  \begin{itemize}
    \item $S$ este o instrucțiune (Stmt)
    \item $Pre$ (precondiție), respectiv $Post$ (postcondiție) sunt aserțiuni logice asupra stării sistemului înaintea, respectiv după execuția lui $S$
  \end{itemize}

  Se asociază fiecărei construcții sintactice Stmt o regulă de deducție care
definește recursiv relația ternară descrisă mai sus.

\begin{block}{Regula Hoare pentru compunerea secvențială}
\[\reg{\{P\}\;c1\, c2\;\{R\}}{\{P\}\;c1 \;\{Q\}\si\{Q\}\;c2\;\{R\}}{}\]
\end{block}
\end{frame}

\begin{frame}{Logica Floyd-Hoare pentru IMP}
\[\reg[Skip]{\{P\}\; \{\} \;\{P\}}{\cdot}{}\]
\vfill
\[\reg[Seq]{\{P\}\;c1 ; c2\;\{R\}}{\{P\}\;c1 \;\{Q\}\si\{Q\}\;c2\;\{R\}}{}\]
\vfill
\[\reg[Asign]{\{P[e/x]\}\;x = e\;\{P\}}{}{}\]
\vfill
\[\reg[If]{\{P\}\Sif c \Sthen t \Selse e\;\{Q\}}{\{c\wedge P\}\;t \;\{Q\}\si\{\neg c\wedge P\}\;e\;\{Q\}}{}\]
\vfill
\[\reg[While]{\{P\}\Swhile(c)\,b\;\{\neg c \wedge P\}}{\{c\wedge P\}\;b \;\{P\}}{}\]
\end{frame}

\begin{section}{Semantică statică}
  \begin{frame}{Semantică Statică - Motivație}
    \begin{itemize}
      \item Este sintaxa unui limbaj de programare prea expresivă?
    \vitem Sunt programe care n-aș vrea să le pot scrie, dar le pot?
    \vitem Putem detecta programe greșite înainte de rulare?

    \onslide<2>De exemplu, în IMP, folosirea variabilelor fără a le declara
    \vitem<2> Soluție: Sistemele de tipuri
    \end{itemize}
    \end{frame}

    \begin{frame}{Sisteme de tipuri}{La ce folosesc?}

    \begin{itemize}
    \item Descriu programele „bine formate“
    \item Pot preveni anumite erori
    \begin{itemize}
    \item folosirea variabilelor nedeclarate/neințializate
    \item detectarea unor bucați de cod inaccesibile
    \item erori de securitate
    \end{itemize}
    \item Ajută compilatorul
    \item Pot influența proiectarea limbajului
    \end{itemize}
    \vfill
    \begin{block}{Scop (ideal)}
    Progamele „bine formate“, i. e., cărora li se poate asocia un tip nu eșuează
    \end{block}
    \end{frame}
  \begin{frame}{Sisteme de tipuri}{Intuiție}
    \begin{itemize}
    \item Vom defini o relație \structure{$\Gamma \vdash e : T$}
    \item Citim \structure{$e$ are tipul $T$ dacă $\Gamma$}, unde
    \item $\Gamma$ — tipuri asociate locațiilor din $e$
    \end{itemize}

    \begin{block}{Exemple}
    \[\begin{array}{lll@{\;:\;}ll}
    &\vdash& \Sif {\Strue} \Sthen \{\} \Selse \{\} & \terminal{stmt}
    \\\\
    x:\Sint&\vdash& x + 13 & \Sint
    \\\\
    x:\Sint&\not\vdash&x = y +1 &T&\mbox{pentru orice } T
    \end{array}\]
    \end{block}
    \end{frame}

    \begin{frame}{Tipuri în limbajul IMP}
    \begin{block}{Tipurile expresiilor = tipurile gramaticale}
    \[T ::= {\Sint} \mid {\Sbool} \mid {\terminal{stmt}}\]
    \end{block}
    \begin{block}{$\Gamma$ --- Mediul de tipuri}
    \begin{itemize}
    \item Asociază tipuri identificatorilor
    \[\Gamma : \mathbb{X} \xrightarrow{\circ}\mathbb{T}\]
    \item \structure{Notație:} o listă de perechi locație-tip

    \[x_1:\Sint, \ldots, x_n:\Sint\]
    \end{itemize}
  \end{block}
    \vfill
    \begin{block}{Observații pentru limbajul IMP}
    \begin{itemize}
    \item Toate locațiile din $\Gamma$ au același tip: $\Sint$
    \item Apariția unei locații în $\Gamma$ înseamnă că locația e de fapt definită
    \end{itemize}
    \end{block}
    \end{frame}

    \begin{frame}{IMP: Reguli pentru tipuri}
    \begin{itemize}
    \vitem[] $\reg[loc]{\tjud{x}{int}}{}{\Gamma(x) = \Sint}$
    \vitem[] $\reg[int]{\tjud{n}{\Sint}}{}{n \in \mathbb{Z}}$
    \vitem[] $\reg[op+]{\tjud{e_1 + e_2}{\Sint}}{\tjud{e_1}{\Sint} \si \tjud{e_2}{\Sint}}{}$
    \vitem[] $\reg[bool]{\tjud{b}{\Sbool}}{}{b\in \{\Strue,\Sfalse\}}$
    \vitem[] $\reg[op$\leq$]{\tjud{e_1 \terminal{<=} e_2}{\Sbool}}{\tjud{e_1}{\Sint} \si \tjud{e_2}{\Sint}}{}$
    \item[] $\reg[atrib]{\tjud{x\terminal{=}e}{\terminal{stmt}}}{\tjud{e}{\Sint}}{\Gamma(x) = \Sint}$
    \vitem[] $\reg[secv]{\tjud{c_1\terminal{;} c_2}{\terminal{stmt}}}{\tjud{c_1}{\terminal{stmt}}\si \tjud{c_2}{\terminal{stmt}}}{}$
    \vitem[] $\reg[if]{\tjud{\Sif c \Sthen t \Selse e}{\terminal{stmt}}}{\tjud{c}{\Sbool} \si \tjud{t}{{\terminal{stmt}}} \si \tjud{e}{\terminal{stmt}}}{}$
    \vitem[] $\reg[while]{\tjud{\Swhile (c)\,b}{\terminal{stmt}}}{\tjud{c}{\Sbool} \si \tjud{b}{{\terminal{stmt}}}}{}$
    \end{itemize}
    \end{frame}

\end{section}


\begin{section}{Semantica operațională}
\begin{frame}{Semantică operațională}
{Plan}
\begin{itemize}
 \item Exemplu de lucru: Limbajul IMP
 \vitem Instrumente de lucru
 \begin{itemize}
  \item Sintaxă, memorie, configurații
  \item Reguli de deducție și arbori de derivare
 \end{itemize}
 \vitem Semantica evaluării
   \begin{itemize}
     \item semantică naturală, într-un pas mare (Big-Step)
	\end{itemize}
 \vitem Semantica tranzițională
	\begin{itemize}
		\item Semantica operațională structurală, a pașilor mici (small-step)
	\end{itemize}
 \end{itemize}
\end{frame}

\begin{frame}{Starea execuției}
\begin{block}{}
Starea execuției unui program IMP la un moment dat este data de valorile deținute în acel moment de variabilele declarate de program.

\structure{Matematic:} o funcție \alert{parțială} $\sigma : \Id \xrightarrow{\circ}\SInt$ de domeniu finit.
\end{block}
\begin{block}{Notații}
\begin{itemize}
\item Descrierea funcției prin enumerare:
$\sigma = n \mapsto 10, sum \mapsto 0$
\item Funcția vidă $\bottom$, nedefinită pentru nici o variabilă
\item Obținerea valorii unei variabile: $\sigma(x)$
\item Suprascrierea valorii unei variabile:

$$\sigma[v/x] (y) = \left\{\begin{array}{r@{\mbox{, dacă }}l}
\sigma	(y) & y \neq x \\
v & y = x
\end{array}
\right.$$
\end{itemize}
\end{block}
\end{frame}

\begin{subsection}{Semantica Evaluării}
\begin{frame}{Semantica Evaluării}
\begin{itemize}
\item Introdusă în 1987 de Gilles Kahn sub numele de „semantică naturală”
\item Denumiri alternative: „semantică relațională”, „semantica big-step”
\item Relaționează fragmente de program într-o stare  cu valoarea corespunzătoare evaluării lor în acea stare
\begin{itemize}
\item Expresiile aritmetice se evaluează la întregi: $\s{\c{a,\sigma}}{\c{i}}$
\item Expresiile Booleene se evaluează la $\Strue$/$\Sfalse$: $\s{\c{b,\sigma}}{\c{t}}$
\item Instrucțiunile se evaluează la stări: $\s{\c{s,\sigma}}{\c{\sigma'}}$
\item Blocurile se evaluează la stări: $\s{\c{{\it bl},\sigma}}{\c{\sigma'}}$
\item Programul se evaluează la o stare: $\s{\c{p}}{\c{\sigma}}$ sau $\s{\c{p,\sigma}}{\c{\sigma'}}$
\end{itemize}
\item Valoarea este obținută într-un \structure{singur pas (mare)}
\item Reguli structurale, având ca premize secvenți corespunzători subtermenilor
\end{itemize}

\begin{block}{Exemple}
\begin{itemize}
\item $\s{\c{3 + x, (x\mapsto 5, y\mapsto 7)}}{\c{8}}$
\item $\s{\c{x = 3 + y, (x\mapsto 5, y\mapsto 7)}}{\c{x\mapsto 10, y\mapsto 7}}$
\end{itemize}
\end{block}
\end{frame}

\begin{frame}{Semantica Evaluării a lui IMP}
{Expresii aritmetice}
\begin{itemize}
\item[] $\reg[Int]{\S{\c{i,\sigma}}{\c{i}}}{}{}$
\vitem[] $\reg[Id]{\S{\c{x,\sigma}}{\c{i}}}{}{i = \sigma(x)}$
\vitem[] $\reg[Add]{\S{\c{a_1 + a_2,\sigma}}{\c{i}}}{\S{\c{a_1,\sigma}}{\c{i_1}}\si\S{\c{a_2,\sigma}}{\c{i_2}}}{i = i_1 +_{\SInt} i_2}$
\vitem[] $\reg[Mul]{\S{\c{a_1 * a_2,\sigma}}{\c{i}}}{\S{\c{a_1,\sigma}}{\c{i_1}}\si\S{\c{a_2,\sigma}}{\c{i_2}}}{i = i_1 *_{\SInt} i_2}$
\vitem[] $\reg[Div]{\S{\c{a_1 / a_2,\sigma}}{\c{i}}}{\S{\c{a_1,\sigma}}{\c{i_1}}\si\S{\c{a_2,\sigma}}{\c{i_2}}}{i_2 \neq 0 \wedge i = i_1 /_{\SInt} i_2}$
\end{itemize}
\end{frame}
\begin{frame}{Semantica Evaluării a lui IMP}
{Expresii booleene}
\begin{itemize}
\item[] $\reg[Bool]{\S{\c{t,\sigma}}{\c{t}}}{}{}$
\vitem[] $\reg[Cmp]{\S{\c{a_1 \terminal{<=} a_2,\sigma}}{\c{t}}}{\S{\c{a_1,\sigma}}{\c{i_1}}\si\S{\c{a_2,\sigma}}{\c{i_2}}}{t = i_1 \leq_{\SInt} i_2}$
\vitem[] $\reg[Not-True]{\S{\c{\Snot b,\sigma}}{\c{\Strue}}}{\S{\c{b,\sigma}}{\c{\Sfalse}}}{}$
\hfill
$\reg[Not-False]{\S{\c{\Snot b,\sigma}}{\c{\Sfalse}}}{\S{\c{b,\sigma}}{\c{\Strue}}}{}$
\vitem[] $\reg[And-True]{\S{\c{b_1 \terminal{\&\&} b_2,\sigma}}{\c{t}}}{\S{\c{b_1,\sigma}}{\c{\Strue}}\si\S{\c{b_2,\sigma}}{\c{t}}}{}$
\vitem[] $\reg[And-False]{\S{\c{b_1 \terminal{\&\&} b_2,\sigma}}{\c{\Sfalse}}}{\S{\c{b_1,\sigma}}{\c{\Sfalse}}}{}$
\end{itemize}
\end{frame}
\begin{frame}{Semantica Evaluării a lui IMP}
{Instrucțiuni simple}
\begin{itemize}
\item[] $\reg[Secv]{\S{\c{s_1 \mathrel{} s_2,\sigma}}{\c{\sigma''}}}{\S{\c{s_1,\sigma}}{\c{\sigma'}}\si\S{\c{s_2,\sigma'}}{\c{\sigma''}}}{}$
\vitem[] $\reg[Asgn]{\S{\c{x \terminal{=} a \terminal{;}, \sigma}}{\c{\sigma'}}}{\S{\c{a,\sigma}}{\c{i}}}{\sigma' = \sigma[i/x]}$
\vitem[] $\reg[If-True]{\S{\c{\Sif\;(b)\;{\it bl}_1\Selse {\it bl}_2,\sigma}}{\c{\sigma_1}}}{\S{\c{b,\sigma}}{\c{\Strue}}\si \S{\c{{\it bl}_1,\sigma}}{\c{\sigma_1}}}{}$
\vitem[] $\reg[If-False]{\S{\c{\Sif\;(b)\;{\it bl}_1\Selse {\it bl}_2,\sigma}}{\c{\sigma_2}}}{\S{\c{b,\sigma}}{\c{\Sfalse}}\si \S{\c{{\it bl}_2,\sigma}}{\c{\sigma_2}}}{}$
\end{itemize}
\end{frame}
\begin{frame}{Semantica Evaluării a lui IMP}
{Blocuri și instrucțiuni de ciclare}
\begin{itemize}
\item[] $\reg[Skip]{\S{\c{\terminal{\{\}},\sigma}}{\c{\sigma}}}{}{}$
\vitem[] $\reg[Block]{\S{\c{\terminal{\{}s\terminal{\}},\sigma}}{\c{\sigma'}}}{\S{\c{s,\sigma}}{\c{\sigma'}}}{}$
\vitem[] $\reg[While-True]{\S{\c{\Swhile\;(b)\;{\it bl},\sigma}}{\c{\sigma''}}}{\S{\c{b,\sigma}}{\c{\Strue}}\si \S{\c{{\it bl},\sigma}}{\c{\sigma'}} \si \S{\c{\Swhile\;(b)\;{\it bl},\sigma'}}{\c{\sigma''}}}{\hspace{-2.5em}}$
\vitem[] $\reg[While-False]{\S{\c{\Swhile\;(b)\;{\it bl},\sigma}}{\c{\sigma}}}{\S{\c{b,\sigma}}{\c{\Sfalse}}}{}$
\end{itemize}
\end{frame}
\begin{frame}{Semantica Evaluării a lui IMP}
{Inițializări; Semantica programului}
\begin{itemize}
\item[] $\reg[Init]{\S{\c{\Sint x \terminal{=} i \terminal{;} p, \sigma}}{\c{\sigma''}}}{\S{\c{p,\sigma'}}{\c{\sigma''}}}{\sigma' = \sigma[i/x]}$
\vitem[] $\reg[Pgm]{\S{\c{p}}{\c{\sigma}}}{\S{\c{p,\bottom}}{\c{\sigma}}}{}$
\end{itemize}
\end{frame}
\begin{frame}{Semantica Evaluării a lui IMP}
{Arbori de derivare}

$\reg[Pgm]{
  \S{\c{\terminal{int} a \terminal{=} \terminal{3} \terminal{;} a \terminal{=} a \terminal{+} 4 \terminal{;}}}{\c{\alt<4->{a\mapsto \alt<5->{7}{??}}{?}}}
 }{
   \only<2->{\reg[Init]{
     \S{\c{\terminal{int} a \terminal{=} \terminal{3} \terminal{;} a \terminal{=} a \terminal{+} 4 \terminal{;},\bottom}}{\c{\alt<4->{a\mapsto \alt<5->{7}{??}}{?}}}
   }{
     \only<3->{\reg[Asgn]{
	   \S{\c{a \terminal{=} a \terminal{+} 4 \terminal{;},a\mapsto 3}}{\c{\alt<4->{a\mapsto \alt<5->{7}{??}}{?}}}
	 }{
	   \only<4->{\reg[Add]{
	     \S{\c{a + 4,a \mapsto 3}}{\c{\alt<5->{7}{??}}}
	   }{
	     \only<5->{\reg[Id]{
		   \S{\c{a, a\mapsto 3}}{\c{3}}
		 }{
		 }{}\si \reg[Int]{
		   \S{\c{4, a\mapsto 3}}{\c{4}}
		 }{}{}}
	   }{}}
	 }{}}}{}}
	 }{}$
\end{frame}
\begin{frame}{Semantica Evaluării}
{Avantaje și dezavantaje}
\begin{block}{Avantaje}
\begin{itemize}
\item Compozitională: arborii de demonstrații sunt compoziționali
\item Ușor și relativ eficient de implementat și executat
\item Foarte folosită pentru definirea sistemelor de tipuri
\end{itemize}
\end{block}
\begin{alertblock}{Dezavantaje}
\begin{itemize}
\item Lipsa granularității --- computația e un monolit
\item Greu de capturat nedeterminismul/concurența
\item Greu de capturat schimbările de control
\item Nemodulară - extensiile solicită modificarea regulilor existente.
\end{itemize}
\end{alertblock}
\end{frame}

\end{subsection}

\end{section}


\begin{section}{Semantica tranzițională (Small-Step SOS)}
  \begin{frame}{Semantica Tranzițională}
  \begin{itemize}
  \item Introdusă de Gordon Plotkin (1981) ca \structure{S}emantică \structure{O}perațională \structure{S}tructurală
  \item Denumiri alternative: „semantică prin tranziții”, „semantică prin reducere"
  \item Definește cel mai mic pas de execuție ca o relație „de tranziție” între configurații:
  $$\Ss{\c{{\it Cod}, {\it Stare}}}{\c{{\it Cod'}, {\it Stare'}}}$$
  \item Fiecare pas de execuție este concluzia unei demonstrații
  \item Execuția se obține ca o succesiune de astfel de tranziții:
  $$\begin{array}{l}\Ss{\c{\alert{\Sint x \terminal{=} 0 \terminal{;}}  x \terminal{=} x \terminal{+} 1 \terminal{;},\bot}}{\Ss{\c{x \terminal{=} \alert{x} \terminal{+} 1 \terminal{;},x \mapsto 0}}{}}
  \\
  \Ss{\c{x \terminal{=} \alert{0 \terminal{+} 1} \terminal{;},x \mapsto 0}}{\Ss{\c{\alert{x \terminal{=} 1 \terminal{;}},x \mapsto 0}}{{\c{\terminal{\{}\terminal{\}},x \mapsto 1}}}}
  \end{array}$$
  \end{itemize}
  \end{frame}

  \begin{frame}{Redex. Reguli structurale. Axiome}
  \begin{block}{\alert{Ex}presie \alert{red}uctibilă --- \alert{redex}}
  Reprezintă fragmentul de sintaxă care va fi modificat la următorul pas.

  \hfill$\Sif (0 \terminal{<=} 5 \terminal{+} 7 \terminal{*} \alert<2->{x}) \terminal{\{} r \terminal{=}1\terminal{;} \terminal{\}}\ \Selse\ \terminal{\{} r \terminal{=}0\terminal{;} \terminal{\}} $\hfill\
  \end{block}

  \begin{block}{Reguli structurale --- Folosesc la identificarea următorului redex}
  \begin{itemize}
  \item Definite recursiv pe structura termenilor
  \end{itemize}
  \onslide<3->{
  $$\reg{
   \Ss{\c{\Sif (\alert{b})\; {\it bl}_1 \Selse {\it bl}_2,\sigma}}{\c{\Sif (\structure{b'})\; {\it bl}_1 \Selse {\it bl}_2,\sigma}}
  }{
    \Ss{\c{b,\sigma}}{\c{b',\sigma}}
  }
  {}$$
  }
  \end{block}
  \begin{block}{Axiome --- Realizează pasul computațional}
  \onslide<4>{$$\reg{\Ss{\c{\Sif (\terminal{true})\; {\it bl}_1 \Selse {\it bl}_2,\sigma}}{\c{{\it bl}_1,\sigma}}}{}{}$$}
  \end{block}
  \end{frame}
  \end{section}
  \begin{section}{Semantica SOS a lui IMP}
\begin{comment}
  \begin{subsection}{Sintaxa și starea memoriei}
  \begin{frame}{Sintaxa BNF a limbajului IMP}
  \vspace{-5ex}\begin{syntaxBlock}{\AExp}
  \alert{
  \begin{itemize}
  \item[]\renewcommand{\syntaxKeyword}{}
  \syntax{\Int\Smid\Id}{}
  \syntaxCont{\AExp\terminal{+}\AExp\Smid\AExp\terminal{*}\AExp\Smid\AExp\terminal{/}\AExp}{}
  \item[]\renewcommand{\defSort}{\BExp}
  \syntax{\Bool}{}
  \syntaxCont{\AExp\terminal{<=}\AExp}{}
  \syntaxCont{\Snot\BExp \Smid \BExp \terminal{\&\&} \BExp}{}
  \item[]\renewcommand{\defSort}{\Block}
  \syntax{\terminal{\{}\Stmt\terminal{\}} \Smid \terminal{\{}\terminal{\}}}{}
  \item[]\renewcommand{\defSort}{\Stmt}
  \syntax{\Block\Smid \Stmt\terminal{}\Stmt}{}
  \syntaxCont{\Id\terminal{=}\AExp\terminal{;}}{}
  \syntaxCont{\Sif\;\terminal{(}\BExp\terminal{)}\Block\Selse\Block}{}
  \syntaxCont{\Swhile\;\terminal{(}\BExp\terminal{)}\Block}{}
  \item[]\renewcommand{\defSort}{\Pgm}
  \syntax{\Sint \Id \terminal{=} \Int \terminal{;} \Pgm\Smid \Stmt}{}
  \end{itemize}
  }
  \end{syntaxBlock}
  \end{frame}

  \begin{frame}{Starea execuției}
  \begin{block}{}
  Starea execuției unui program IMP la un moment dat este data de valorile deținute în acel moment de variabilele declarate de program.

  \structure{Matematic:} o funcție \alert{parțială} $\sigma : \Id \xrightarrow{\circ}\SInt$ de domeniu finit.
  \end{block}
  \begin{block}{Notații}
  \begin{itemize}
  \item Descrierea funcției prin enumerare:
  $\sigma = n \mapsto 10, sum \mapsto 0$
  \item Funcția vidă $\bottom$, nedefinită pentru nici o variabilă
  \item Obținerea valorii unei variabile: $\sigma(x)$
  \item Suprascrierea valorii unei variabile:

  $$\sigma[v/x] (y) = \left\{\begin{array}{r@{\mbox{, dacă }}l}
  \sigma	(y) & y \neq x \\
  v & y = x
  \end{array}
  \right.$$
  \end{itemize}
  \end{block}
  \end{frame}
  \end{subsection}
\end{comment}
  \begin{frame}{Semantica SOS a lui IMP}
  {Expresii aritmetice}
  \begin{itemize}
  \item Un întreg este valoare --- nu poate fi redex, deci nu avem regulă
  %\item[] {\color{black!10}$\reg[Int]{\Ss{\c{i,\sigma}}{\c{i,\sigma}}}{}{}$}
  \vitem[] $\reg[Id]{\Ss{\c{x,\sigma}}{\c{i, \sigma}}}{}{i = \sigma(x)}$
  \vitem[] $\reg{\Ss{\c{a_1 + a_2,\sigma}}{\c{a_1' + a_2,\sigma}}}{\Ss{\c{a_1,\sigma}}{\c{a_1',\sigma}}}{}$\hfill$\reg{\Ss{\c{a_1 + a_2,\sigma}}{\c{a_1 + a_2',\sigma}}}{\Ss{\c{a_2,\sigma}}{\c{a_2',\sigma}}}{}$
  \vitem Ordine nespecificată de evaluare a argumentelor
  \vitem[] $\reg[Add]{\Ss{\c{i_1 + i_2,\sigma}}{\c{i,\sigma}}}{}{i = i_1 + i_2}$
  \vitem Regula pentru înmulțire este la fel
  \end{itemize}
  \end{frame}
  \begin{frame}{Semantica SOS a lui IMP}
  {Ordine de evaluare. Împărțire}
  \begin{block}{Semantica împărțirii}
  Evaluăm al doilea argument, și dacă e diferit de 0, atunci evaluăm si primul argument și apoi împărțirea.
  \end{block}
  \begin{itemize}
  \item[] $\reg{\Ss{\c{a_1 / a_2,\sigma}}{\c{a_1 / a_2',\sigma}}}{\Ss{\c{a_2,\sigma}}{\c{a_2',\sigma}}}{}$
  \vitem[] $\reg{\Ss{\c{a_1 / i_2,\sigma}}{\c{a_1' / i_2,\sigma}}}{\Ss{\c{a_1,\sigma}}{\c{a_1',\sigma}}}{i_2 \neq 0}$
  \vitem Evaluarea argumentelor de la dreapta la stânga
  \vitem[] $\reg[Div]{\Ss{\c{i_1 / i_2,\sigma}}{\c{i,\sigma}}}{}{i_2\neq 9, i = i_1 / i_2}$
  \end{itemize}
  \end{frame}
  \begin{frame}{Semantica SOS a lui IMP}
  {Expresii Booleene. Constante și operatorul de comparație.}
  \begin{itemize}
  \item Constantele Booleene sunt valori --- nu pot fi redex
  \vitem[] $\reg{\Ss{\c{a_1 \terminal{<=} a_2,\sigma}}{\c{a_1' \terminal{<=} a_2,\sigma}}}{\Ss{\c{a_1,\sigma}}{\c{a_1',\sigma}}}{}$\hfill$\reg{\Ss{\c{a_1 \terminal{<=} a_2,\sigma}}{\c{a_1 \terminal{<=} a_2',\sigma}}}{\Ss{\c{a_2,\sigma}}{\c{a_2',\sigma}}}{}$
  \vitem[] $\reg[Leq-false]{\Ss{\c{i_1 \terminal{<=} i_2,\sigma}}{\c{\terminal{false},\sigma}}}{}{i_1 > i_2}$
  \vitem[] $\reg[Leq-true]{\Ss{\c{i_1 \terminal{<=} i_2,\sigma}}{\c{\terminal{true},\sigma}}}{}{i_1 \leq i_2}$
  \end{itemize}
  \end{frame}
  \begin{frame}{Semantica SOS a lui IMP}
  {Expresii Booleene. Negația logică}
  \begin{itemize}
  \item[] $\reg{\Ss{\c{\terminal{!} a,\sigma}}{\c{\terminal{!} a',\sigma}}}{\Ss{\c{a,\sigma}}{\c{a',\sigma}}}{}$
  \vitem[] $\reg[!-true]{\Ss{\c{\terminal{!} \terminal{true},\sigma}}{\c{\terminal{false},\sigma}}}{}{}$
  \vitem[] $\reg[!-false]{\Ss{\c{\terminal{!} \terminal{false},\sigma}}{\c{\terminal{true},\sigma}}}{}{}$
  \end{itemize}
  \end{frame}
  \begin{frame}{Semantica SOS a lui IMP}
  {Expresii Booleene. Și-ul logic}
  \begin{itemize}
  \item[] $\reg{\Ss{\c{b_1 \terminal{\&\&} b_2,\sigma}}{\c{b_1' \terminal{\&\&} b_2,\sigma}}}{\Ss{\c{b_1,\sigma}}{\c{b_1',\sigma}}}{}$
  \vitem[] $\reg[\&\&-false]{\Ss{\c{\terminal{false}\; \terminal{\&\&} b_2,\sigma}}{\c{\terminal{false},\sigma}}}{}{}$
  \vitem[] $\reg[\&\&-true]{\Ss{\c{\terminal{true}\; \terminal{\&\&} b_2,\sigma}}{\c{b_2,\sigma}}}{}{}$
  \end{itemize}
  \end{frame}
  \begin{frame}{Semantica SOS a lui IMP}
  {Blocuri}
  \begin{itemize}
  \item Blocul vid $\terminal{\{\}}$ este „valoarea” blocurilor și instrucțiunilor
  \vitem O instrucțiune poate modifica starea curentă

  \[\reg{\Ss{\c{\terminal{\{} s \terminal{\}},\sigma}}{\c{\terminal{\{} s' \terminal{\}},\alert{\sigma'}}}}{\Ss{\c{s,\sigma}}{\c{s',\alert{\sigma'}}}}{}\]
  \vitem[] \[\reg[Block-end]{\Ss{\c{\terminal{\{} \terminal{\{\}} \terminal{\}},\sigma}}{\c{\terminal{\{\}},\sigma}}}{}{}\]
  \end{itemize}
  \end{frame}

  \begin{frame}{Semantica SOS a lui IMP}
  {Compunerea secvențială}
  \begin{itemize}
  \vitem[] $\reg{\Ss{\c{s_1 \terminal{} s_2},\sigma}{\c{s_1' \terminal{} s_2,\sigma'}}}{\Ss{\c{s_1,\sigma}}{\c{s_1',\sigma'}}}{}$
  \vitem[] $\reg[Next-stmt]{\Ss{\c{\terminal{\{\}} \terminal{} s_2,\sigma}}{\c{s_2,\sigma}}}{}{}$
  \end{itemize}
  \end{frame}

  \begin{frame}{Semantica SOS a lui IMP}
  {Atribuirea}
  \begin{itemize}
  \item[] $\reg{\Ss{\c{x \terminal{=} a \terminal{;},\sigma}}{\c{x \terminal{=} a' \terminal{;},\sigma}}}{\Ss{\c{a,\sigma}}{\c{a',\sigma}}}{}$
  \vitem[] $\reg[Asgn]{\Ss{\c{x \terminal{=} i \terminal{;},\sigma}}{\c{\terminal{\{\}},\sigma'}}}{}{\sigma'=\sigma[i/x]}$
  \end{itemize}
  \end{frame}

  \begin{frame}{Semantica SOS a lui IMP}
  {Condițional}
  \begin{itemize}
  \item[]
  $\reg{
   \Ss{\c{\Sif ({b})\; {\it bl}_1 \Selse {\it bl}_2,\sigma}}{\c{\Sif ({b'})\; {\it bl}_1 \Selse {\it bl}_2,\sigma}}
  }{
    \Ss{\c{b,\sigma}}{\c{b',\sigma}}
  }
  {}$
  \vitem[]
  $\reg[If-true]{\Ss{\c{\Sif (\terminal{true})\; {\it bl}_1 \Selse {\it bl}_2,\sigma}}{\c{{\it bl}_1,\sigma}}}{}{}$
  \vitem[]
  $\reg[If-false]{\Ss{\c{\Sif (\terminal{false})\; {\it bl}_1 \Selse {\it bl}_2,\sigma}}{\c{{\it bl}_2,\sigma}}}{}{}$

  \end{itemize}
  \end{frame}

  \begin{frame}{Semantica SOS a lui IMP}
  {Instrucțiunea de ciclare}
  \begin{itemize}
  \item[]
  $\reg[While]{
   \Ss{\c{\Swhile ({b})\; {\it bl},\sigma}}{\c{\Sif ({b})\; \terminal{\{} {\it bl}\terminal{} \Swhile ({b})\; {\it bl} \terminal{\}} \Selse \terminal{\{\}},\sigma}}
  }{}
  {}$
  \end{itemize}
  \end{frame}

  \begin{frame}{Semantica SOS a lui IMP}
  {Inițializări}
  \begin{itemize}
  \item[] $\reg[Init]{\Ss{\c{\terminal{int} x \terminal{=} i \terminal{;} p,\sigma}}{\c{p,\sigma'}}}{}{\sigma'=\sigma[i/x]}$
  \end{itemize}
  \end{frame}

  \begin{frame}{Semantica SOS a lui IMP}
  {Demonstrarea unui pas. Execuție.}

  \begin{itemize}
  \item Fiecare pas de deducție este o demonstrație liniară alcătuită din mai multe reguli structurale și având la vârf o axiomă

  \vitem Execuția este o succesiune de astfel de stări
  \end{itemize}
  \end{frame}

  \begin{frame}{Semantica SOS a lui IMP}
  {Execuție pas cu pas}

  $\begin{array}{lr}\c{\alert{\Sint i \terminal{=} 3 \terminal{;} \Swhile (0 \terminal{<=} i)\; \terminal{\{} i \terminal{=} i \terminal{+} -4 \terminal{;} \terminal{\}}},\bot}&\xrightarrow{\textsc{Init}}\\
  \onslide<2->
  \c{\alert{\Swhile (0 \terminal{<=} i)\; \terminal{\{} i \terminal{=} i \terminal{+} -4 \terminal{;} \terminal{\}}},i \mapsto 3}&\xrightarrow{\textsc{While}}\\
  \onslide<3->
  \c{\Sif (0 \terminal{<=} \alert{i})\begin{array}[t]{l} \terminal{\{}\terminal{\{} i \terminal{=} i \terminal{+} -4 \terminal{;} \terminal{\}} \\ \ \ \ \Swhile (0 \terminal{<=} i)\; \terminal{\{} i \terminal{=} i \terminal{+} -4 \terminal{;} \terminal{\}}\\ \terminal{\}}\; \Selse\; \terminal{\{\}}\end{array},i \mapsto 3}&\xrightarrow{\textsc{Id}}\\
  \onslide<4->
  \c{\Sif (\alert{0 \terminal{<=} 3})\begin{array}[t]{l} \terminal{\{}\terminal{\{} i \terminal{=} i \terminal{+} -4 \terminal{;} \terminal{\}} \\ \ \ \ \Swhile (0 \terminal{<=} i)\; \terminal{\{} i \terminal{=} i \terminal{+} -4 \terminal{;} \terminal{\}}\\ \terminal{\}}\; \Selse\; \terminal{\{\}}\end{array},i \mapsto 3}&\xrightarrow{\textsc{Leq-true}}\\
  \onslide<5->
  \c{\alert{\Sif (true)\begin{array}[t]{l} \terminal{\{}\terminal{\{} i \terminal{=} i \terminal{+} -4 \terminal{;} \terminal{\}} \\ \ \ \ \Swhile (0 \terminal{<=} i)\; \terminal{\{} i \terminal{=} i \terminal{+} -4 \terminal{;} \terminal{\}}\\ \terminal{\}}\; \Selse\; \terminal{\{\}}\end{array}},i \mapsto 3}&\xrightarrow{\textsc{If-true}}\\
  \onslide<6->
  \c{\terminal{\{}\terminal{\{} i \terminal{=} \alert{i} \terminal{+} -4 \terminal{;} \terminal{\}} \terminal{} \Swhile (0 \terminal{<=} i)\; \terminal{\{} i \terminal{=} i \terminal{+} -4 \terminal{;} \terminal{\}} \terminal{\}},i \mapsto 3}&\xrightarrow{\textsc{Id}}\\
  \end{array}$
  \end{frame}
  \begin{frame}{Semantica SOS a lui IMP}
  {Execuție pas cu pas}

  $\begin{array}{lr}
  \c{\terminal{\{}\terminal{\{} i \terminal{=} \alert{3 \terminal{+} -4} \terminal{;} \terminal{\}} \terminal{} \Swhile (0 \terminal{<=} i)\; \terminal{\{} i \terminal{=} i \terminal{+} -4 \terminal{;} \terminal{\}} \terminal{\}},i \mapsto 3}&\xrightarrow{\textsc{Add}}\\
  \onslide<2->
  \c{\terminal{\{}\terminal{\{}\alert{ i \terminal{=} -1 \terminal{;}} \terminal{\}} \terminal{} \Swhile (0 \terminal{<=} i)\; \terminal{\{} i \terminal{=} i \terminal{+} -4 \terminal{;} \terminal{\}} \terminal{\}},i \mapsto 3}&\xrightarrow{\textsc{Asgn}}\\
  \onslide<3->
  \c{\terminal{\{}\alert{\terminal{\{} \terminal{\{\}} \terminal{\}}} \terminal{} \Swhile (0 \terminal{<=} i)\; \terminal{\{} i \terminal{=} i \terminal{+} -4 \terminal{;} \terminal{\}} \terminal{\}},i \mapsto -1}&\xrightarrow{\textsc{Block-end}}\\
  \onslide<4->
  \c{\terminal{\{} \alert{\terminal{\{\}} \terminal{} \Swhile (0 \terminal{<=} i)\; \terminal{\{} i \terminal{=} i \terminal{+} -4 \terminal{;} \terminal{\}}} \terminal{\}},i \mapsto -1}&\xrightarrow{\textsc{Next-stmt}}\\
  \onslide<5->
  \c{\terminal{\{}\alert{\Swhile (0 \terminal{<=} i)\; \terminal{\{} i \terminal{=} i \terminal{+} -4 \terminal{;} \terminal{\}}} \terminal{\}},i \mapsto -1}&\xrightarrow{\textsc{While}}\\
  \onslide<6->
  \c{\terminal{\{}\;\Sif (0 \terminal{<=} \alert{i})\begin{array}[t]{l} \terminal{\{}\terminal{\{} i \terminal{=} i \terminal{+} -4 \terminal{;} \terminal{\}} \\ \ \ \ \Swhile (0 \terminal{<=} i)\; \terminal{\{} i \terminal{=} i \terminal{+} -4 \terminal{;} \terminal{\}}\\ \terminal{\}}\; \Selse\; \terminal{\{\}}\terminal{\}}\end{array},i \mapsto -1}&\xrightarrow{\textsc{Id}}\\
  \onslide<7->
  \c{\terminal{\{}\;\Sif (\alert{0 \terminal{<=} -1})\begin{array}[t]{l} \terminal{\{}\terminal{\{} i \terminal{=} i \terminal{+} -4 \terminal{;} \terminal{\}} \\ \ \ \ \Swhile (0 \terminal{<=} i)\; \terminal{\{} i \terminal{=} i \terminal{+} -4 \terminal{;} \terminal{\}}\\ \terminal{\}}\; \Selse\; \terminal{\{\}}\terminal{\}}\end{array},i \mapsto -1}&\xrightarrow{\textsc{Leq-false}}\\

  \end{array}$
  \end{frame}

  \begin{frame}{Semantica SOS a lui IMP}
  {Execuție pas cu pas}

  $\begin{array}{lr}
  \c{\terminal{\{}\;\alert{\Sif (\terminal{false})\begin{array}[t]{l} \terminal{\{}\terminal{\{} i \terminal{=} i \terminal{+} -4 \terminal{;} \terminal{\}} \\ \ \ \ \Swhile (0 \terminal{<=} i)\; \terminal{\{} i \terminal{=} i \terminal{+} -4 \terminal{;} \terminal{\}}\\ \terminal{\}}\; \Selse\; \terminal{\{\}}{\color{black}\terminal{\}}}\end{array}},i \mapsto -1}&\xrightarrow{\textsc{If-false}}\\
  \onslide<2->
  \c{\alert{\terminal{\{} \terminal{\{\}} \terminal{\}}},i \mapsto -1}&\xrightarrow{\textsc{Block-end}}\\
  \onslide<3->
  \c{\terminal{\{\}},i \mapsto -1}
  \end{array}$
  \end{frame}

  \begin{frame}{Semantica SOS}{Avantaje și dezavantaje}
  \begin{block}{Avantaje}
  \begin{itemize}
  \item Definește precis noțiunea de pas computațional
  \item Semnalează erorile, oprind execuția
  \item Execuția devine ușor de urmărit și depanat
  \item Nedeterminismul și concurența pot fi definite și analizate
  \end{itemize}
  \end{block}
  \vfill\begin{alertblock}{Dezavantaje}
  \begin{itemize}
  \item Regulile structurale sunt evidente și deci plictisitor de scris
  \item Schimbarea abruptă a controlului rămâne o o sarcină dificilă
  \item Nemodular: adăugarea unei trăsături noi poate solicita schimbarea întregii definiții
  \end{itemize}
  \end{alertblock}
  \end{frame}

  \end{section}
\end{document}



