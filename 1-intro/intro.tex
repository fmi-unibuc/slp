\documentclass[xcolor=pdftex,romanian,colorlinks]{beamer}

\usepackage{../tslides}
\usepackage[all]{xy}

\title[SLP---Introducere]{Semantica limbajelor de programare}
\subtitle{(bazat pe cursul omonim de la Universitatea Cambridge)}
\begin{document}
\begin{frame}
  \titlepage
\end{frame}

\begin{section}{Motivație}
\begin{frame}{Ce înseamnă semantica formală?}%
{Ce definește un limbaj de programare?}
\begin{description}
\vitem[Sintaxa] Simboluri de operație, cuvinte cheie, descriere (formală) a programelor/expresiilor bine formate
\vitem[Practica] Un limbaj e definit de modul cum poate fi folosit
\begin{itemize}
\item Manual de utilizare și exemple de bune practici
\item Implementare (compilator/interpretor)
\item Instrumente ajutătoare (analizor de sintaxă, depanator)
\end{itemize}  
\vitem[Semantica?] Ce înseamnă / care e comportamentul unei instrucțiuni?
\begin{itemize}
\item De cele mai multe ori se dă din umeri și se spune că \structure{Practica} e suficientă 
\end{itemize}
\end{description}
\end{frame}

\begin{frame}{La ce folosește semantica}
\begin{itemize}
\item Să înțelegem un limbaj în profunzime
\begin{itemize}
\item Ca programator: pe ce mă pot baza când programez în limbajul dat
\item Ca implementator al limbajului: ce garanții trebuie să ofer
\end{itemize}
\vitem Ca instrument în proiectarea unui nou limbaj / a unei extensii
 \begin{itemize}
\item Înțelegerea componentelor și a relațiilor dintre ele
\item Exprimarea (și motivarea)  deciziilor de proiectare
\item Demonstrarea unor proprietăți generice ale limbajului\\
E.g., execuția nu se va bloca pentru programe care trec de analiza tipurilor
\end{itemize}
\vitem Ca bază pentru demonstrarea corectitudinii programelor.
\end{itemize}
\end{frame}

\begin{frame}[fragile]{C}
\onslide<2>
\begin{block}{}
\begin{verbatim}
  int main(void) {
    int x = 3;
    return x++ + x++ + x++ + x++;
  }
\end{verbatim}
\end{block}
\end{frame}

\begin{frame}[fragile]{C}

\begin{block}{}
\begin{verbatim}
  int main(void) {
    int x = 0;
    return (x = 1) + (x = 2);
  }
\end{verbatim}
\end{block}
\onslide<2>
Conform standardului C, comportamentul programului este \structure{nedefinit}.
\begin{itemize}
\item GCC4, MSVC: valoarea întoarsă e \alert{4}
\item GCC3, ICC, Clang: valoarea întoarsă e \alert{3}
\end{itemize}
\end{frame}

\begin{frame}[fragile]{C}

\begin{block}{}
\begin{verbatim}
  int r; 
  int f(int x) {
    return (r = x);
  }
  int main() {
    return f(1) + f(2), r;
  }
\end{verbatim}
\end{block}

\onslide<2> Conform standardului C, comportamentul programului este \structure{subspecificat}:\\
poate întoarce atât valoarea \alert{1} cât și \alert{2}.
\end{frame}

\begin{frame}[fragile]{C\#}
\onslide<2>
\begin{verbatim}
delegate int IntThunk();
class M {
    public static void Main() {
        IntThunk[] funcs = new IntThunk[11];
        for (int i = 0; i <= 10; i++) {
            funcs[i] = delegate() { return i; };
        }
        foreach (IntThunk f in funcs) {
            System.Console.WriteLine(f());
        }
    }
}
\end{verbatim}
\end{frame}


\begin{frame}[fragile]{C++}
\onslide<2>
\begin{verbatim}
#include <iostream>
#include <array>
#include <functional>
using namespace std;

int main() {
    array<function<int()>,11> funcs;
    for (auto i = 0; i < 11; i++) {
    	funcs[i] = [&]() {return i;};
    }
    for (auto f : funcs) {
    	cout << f() << endl;
    }
}
\end{verbatim}
\end{frame}

\begin{frame}[fragile]{JavaScript}
\onslide<2>
\begin{verbatim}
function bar(x) { 
  return function() { var x = 5; return x; }; 
}
var f = bar(200);
f()
\end{verbatim}
\end{frame}

\begin{frame}[fragile]{JavaScript}
\begin{verbatim}
function bar(x) { 
  return function() { var x = x; return x; }; 
}
var f = bar(200);
f()
\end{verbatim}
\end{frame}

\begin{frame}[fragile]{Java}
\onslide<2>
\small{\begin{verbatim}
class Main {
    interface F<A, B> {          B a(A x);         }
    static <A, B, C> F<A, C> c(final F<A, B> f, final F<B, C> g) {
        return new F<A, C>() { 
            public C a(A x) {  return g.a(f.a(x)); }
        };
    }
    public static void main(String[] args) {
        final Integer a = 2, b = 1;
        F<Integer, Integer> f = new F<Integer, Integer>() { 
            public Integer a(Integer x) { return x + b; } };
        F<Integer, Integer> g = new F<Integer, Integer>() { 
            public Integer a(Integer x) { return a * x; } };
        F<Integer, Integer> h = c( f, g );
        System.out.println(h(5));
    }
}
\end{verbatim}}
\end{frame}


\end{section}

\begin{section}{Pe scurt}
\begin{frame}{Descriere curs}
\begin{itemize}
 \item Definirea unui mini-limbaj de programare
	\begin{itemize}
        \item Cu extensii, atât declarative, cât și imperative
	\item Definirea sistemurilor de tipuri
	\end{itemize}
 \vitem Elemente de proiectarea limbajelor de programare
	\begin{itemize}
		\item Tipuri, Funcții, Date structurate, Referințe
		\item Subtipuri, Obiecte, Interacție și Concurență
	\end{itemize}
 \vitem Proprietăți ale limbajelor de programare și ale programelor lor
   \begin{itemize}
        \item Instrumente (simple) matematice pentru studiul programelor
	\item Corectitudinea programelor în raport cu un sistem de tipuri
	\item Echivalența semantică dintre programe
   \end{itemize}
\end{itemize}
\end{frame}

\begin{frame}{Obiectiv}
\begin{itemize}
\item Formalizarea și înțelegerea conceptelor de bază în proiectarea limbajelor de programare
\begin{itemize}
  \item Sintaxă, semantica execuției, semantica tipurilor
\end{itemize}
\end{itemize}
\end{frame}


\begin{frame}{Privire de ansamblu}{}
\xymatrix@C=1em@R=5em{
\mbox{Limbaje formale} \ar[dr]& \mbox{Logică}\ar[d] & \mbox{C/Java/\ldots}\ar[dl] & \mbox{Calculabilitate}\ar[dll]\\
&\mbox{\large \color{blue}Semantica}\ar[dl]\ar[d]\ar[dr]\ar[drr] \\
\mbox{Compilare} & \mbox{Verificare} & \mbox{Concurență} & \mbox{Semantica denotațională} 
}
\end{frame}


\end{section}

\begin{section}{Evaluare}
\begin{frame}{Evaluare}
\begin{itemize}
\item Examen scris: 60-70\%
\vitem Recenzie detaliată a unui articol (la alegere) din track-ul principal la una din conferințele POPL sau ESOP: 30\%
\vitem Prezentarea articolului recenzat la seminar: 10\% 
\vitem Activați-vă (dacă e cazul) contul de Moodle
\begin{itemize}
\item Alegerea articolului pe care îl veți recenza trebuie postată pe Moodle, pentru a evita situația când mai mulți oameni fac recenzie aceluiași articol.
\end{itemize}
\end{itemize}
\end{frame}

\begin{frame}{Examenul scris}
\begin{itemize}
\item Va consta din probleme
\vitem Scopul lui e să verifice fixarea cunoștințelor predate
\vitem Cu acces la materiale tipărite/xeroxate (nu scrise de mână)
\end{itemize}

\begin{block}{Precizare}
Pentru promovare, nota de la examenul scris trebuie să fie minim 5.
\end{block}
\end{frame}
\end{section}

\begin{frame}{Resurse}
\begin{itemize}
\item Pagina Moodle a cursului: \url{http://moodle.fmi.unibuc.ro/course/view.php?id=522}
\begin{itemize}
\item Prezentările cursurilor, forumuri, resurse electronice
\end{itemize}
\vitem Pagina cursului similar de la Cambridge \url{http://www.cl.cam.ac.uk/teaching/1415/Semantics}
\end{itemize}
\end{frame}
\end{document}



