\documentclass[xcolor=pdftex,romanian,colorlinks,handout]{beamer}

\usepackage{../tslides}


\title[SLP---Recenzii]{Cum se face o recenzie științifică}
\begin{document}
\begin{frame}
  \titlepage
\end{frame}


\begin{section}{Alegerea articolului}

\begin{frame}{Proceedings POPL sau ESOP}

\begin{block}{POPL: Principles Of Programming Languages (internațional conferece on)}
\begin{itemize}
\item DBLP: \href{http://dblp.uni-trier.de/search/publ?q=venue\%3APOPL\%3A}{http://dblp.uni-trier.de/search/publ?q=venue\%3APOPL\%3A}
\item Alegeți un articol de cel puțin 10 pagini 

Certifică că a fost recenzat, acceptat și prezentat în track-ul principal al conferinței.

\end{itemize}
\end{block}


\begin{block}{ESOP: European Symposium On Programming}
\begin{itemize}
\item DBLP: \href{http://dblp.uni-trier.de/search/publ?q=venue\%3AESOP\%3A}{http://dblp.uni-trier.de/search/publ?q=venue\%3AESOP\%3A}

\item Alegeți un articol de cel puțin 20 pagini 

Certifică că a fost recenzat, acceptat și prezentat în track-ul principal al conferinței.
\end{itemize}
\end{block}
\end{frame}

\begin{frame}{Cum alegem un articol?}
\begin{itemize}
\item Alegeți un titlu care vi se pare interesant/abordabil
\item Alegeți link-ul `view' (cel mai din stânga) corespunzător articolului  
\item Folosiți abstract-ul pentru a vă hotărî dacă alegerea e bună
\end{itemize}
\end{frame}

\begin{frame}{Cum obținem un articol?}
\begin{itemize}
    \item Căutați numele articolului pe Google / Google Scholar
     \begin{itemize}
      \item De multe ori se găsește gratis pe situl unuia din autori
     \end{itemize}
	\item Folosiți portalul ANELIS: \href{http://anelis1.summon.serialssolutions.com/}{http://anelis1.summon.serialssolutions.com/}
	\begin{itemize}
	\item Alegeți link-ul Resurse electronice
	\item Pentru ESOP alegeți SpringerLink
	\end{itemize}
\end{itemize}
\end{frame}

\end{section}

\end{document}



