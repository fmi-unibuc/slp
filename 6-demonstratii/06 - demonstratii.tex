\documentclass[xcolor=pdftex,romanian,colorlinks,handout]{beamer}

\usepackage{../tslides}


\title[SLP---Demonstrații]{Demonstrații II}
\begin{document}
\begin{frame}
  \titlepage
\end{frame}


\begin{section}{Inducție}

\begin{frame}{Inducție deductivă pentru mulțimi definite recursiv}

\begin{block}{Principiul inducției deductive}
Pentru a demonstra că $P$ este adevărată pentru orice element al unei mulțimi $A$ definită recursiv de regulile din $\cal R$:
\begin{itemize}
  \item Considerăm mulțimea elementelor din $A$ pentru care $P$ e adevărată
 \item Arătăm că este închisă la regulile din $\cal R$
\end{itemize}

Concret, pentru fiecare regulă $(H,c)$, cu \structure{$H\subseteq A$}
\begin{description} 
\item[Ipoteza de inducție] Presupunem că $P(h)$ e adevărată pentru orice $h\in H$,
\item[Concluzie] Demonstrăm că $P(c)$ e adevărată.
\end{description}	
\end{block}
\end{frame}


\begin{frame}{Arbori de sintaxă}
\begin{block}{Sintaxă}
\alert{
\begin{syntaxBlock}{\Exp}
\renewcommand{\syntaxKeyword}{}
\syntax{\Int\Smid\Bool \Smid {\Exp}\op {\Exp} \Smid\Sif\Exp\Sthen \Exp\Selse\Exp  }{}
\syntaxCont{\terminal{!}\Loc \Smid \Loc \terminal{:=} \Exp}{}
\syntaxCont{\Sskip \Smid  \Exp\terminal{;}\Exp \Smid \Swhile \Exp\Sdo \Exp \Sdone}{}
\end{syntaxBlock}
}
\end{block}

\begin{block}{Univers}
Totalitatea arborilor cu etichete din mulțimea:
\[\begin{array}{r@{\;}l}\mathbb{E} = & \{n\mid n\in\mathbb{Z}\} \cup \{\Strue, \Sfalse\}\cup \{\_+\_, \_\leq\_\} \cup \{\Sif\_\Sthen\_\Selse\_\}
\\ 
\cup &
\{! l \mid l\in  \mathbb{L}\}\cup \{l :=\_\mid l\in \mathbb{L}\}
\\
\cup &
\{\Sskip, \_;\_, \Swhile\_\Sdo\_\Sdone\}
\end{array}\]
\end{block}

\begin{block}{Arbori abstracți de sintaxă}
Definiți recursiv ca submulțime a arborilor de mai sus
\end{block}
\end{frame}

\begin{frame}
{Reguli de formare a arborilor abstracți de sintaxă}

$\reg[sN]{\displaystyle n}{}{n\in \mathbb{Z}}$
\hfill $\reg[sB]{\displaystyle b}{}{b\in \{\Strue, \Sfalse\}}$

\vfill$\reg[sSkip]{\displaystyle \Sskip}{}{}$
\hfill$\reg[sLoc]{\displaystyle !l}{}{l\in \mathbb{L}}$

\vfill $\reg[sAtrib]{\mbox{\arbore{l := \_}{}{e}{}}}{\displaystyle e}{l\in \mathbb{L}}$

\vfill$\reg[sOp]{\mbox{\arbore{$\_\mathrel{o}\_$}{$e_1$}{}{$e_2$}}}{\displaystyle e_1 \si e_2}{o \in \{+,\leq\}}$


\vfill$\reg[sSecv]{\mbox{\arbore{\_;\_}{$e_1$}{}{$e_2$}}}{\displaystyle e_1 \si e_2}{}$
\hfill $\reg[sWhile]{\mbox{\arbore{$\Swhile\_\Sdo\_\Sdone$}{$e_1$}{}{$e_2$}}}{\displaystyle e_1 \si e_2}{}$

 \vfill $\reg[sIf]{\mbox{\arbore{$\Sif\_\Sthen\_\Selse\_$}{$e_1$}{$e_2$}{$e_3$}}}{\displaystyle e_1 \si e_2 \si e_3}{}$\hfill\;

\end{frame}


\begin{frame}{Principiul inducției structurale}{Inducție pe termeni definiți recursiv}
\begin{block}{Scop:} 
Demonstrăm că $P$ este adevărată pentru orice termen (AST) definit (recursiv) de o gramatică independentă de context. 
\end{block}

\begin{block}{Metoda:} 
Instanțiem principiul inducției deductive pentru sistemul de reguli indus de producțiile gramaticii.
\end{block}

\begin{block}{Concret} 
Pentru fiecare regulă de formare a expresiilor indusă de gramatică, 
\begin{description} 
\item[Ipoteza de inducție]  Presupunem că $P$ ține pentru subexpresiile componente
\item[Concluzie] Demonstrăm că $P$ ține și pentru expresia definită de regulă.
\end{description}	
\end{block}

\end{frame}

\begin{frame}<handout:0>{Cazuri de inducție structurală pentru IMP}
\begin{itemize}
\item[]\structure{Cazuri de bază.}  
Demonstrăm
\begin{itemize}
\item $P(n)$ pentru orice $n \in \mathbb{Z}$,
\item $P(\Strue)$ și $P(\Sfalse)$
\item $P(\Sskip)$
\item$P(\terminal{!} l)$ pentru orice $l\in \mathbb{L}$
\end{itemize}

\item[] \structure{Recursie simplă.}  
Demonstrăm că dacă $P(e)$ atunci și
\begin{itemize}
\item $P(l \terminal{:=} e)$ pentru orice $l \in \mathbb{L}$
\end{itemize}
\item[] \structure{Recursie dublă.}
Demonstrăm că dacă $P(e_1)$ și $P(e_2)$, atunci și
\begin{itemize}
\item $P(e_1 \terminal{o} e_2)$ pentru orice $o \in \{\leq, +\}$
\item $P(e_1 \terminal{;} e_2)$
\item $P(\Swhile e_1 \Sdo e_2 \Sdone)$ 
\end{itemize}
\item[]\structure{Recursie triplă.}
Demonstrăm că dacă $P(e_1)$, $P(e_2)$ și $P(e_3)$, atunci și 
\begin{itemize}
\item $P(\Sif e_1 \Sthen e_2 \Selse e_3)$ 
\end{itemize}
\end{itemize}
\end{frame}

\end{section}


\begin{section}{Demonstrații}


\begin{frame}{Ce este o demonstrație?}
\begin{block}{O demonstrație formală}
\begin{itemize}
\item O derivare a concluziei din premize folosind logică formală
\item Exemplu: un arbore de demonstrație
\item Greu de scris de mână dar ușor de verificat automat
\end {itemize}
\end{block}
\begin{block}{O demonstrație informală, dar riguroasă}
\begin{itemize}
\item Noțiunea uzuală de demonstrație matematică
\item Un argument cu \structure{suficiente} detalii pentru a convinge că poate fi transformat într-o demonstrație formală
\end {itemize}
\end{block}
\begin{block}{Apă de ploaie}
Orice nu se încadrează în cele două cazuri de mai sus.
\end {block}
\end{frame}

\begin{frame}{De ce demonstrăm lucruri „evidente“?}
\begin{itemize}
\item<1-> O demonstrație ne poate arăta \structure{de ce} e evidentă o afirmație
\vitem<2-> Uneori afirmațiile evidente se dovedesc a fi \alert{false}
\begin{itemize}
\item<3-> O demonstrație ne poate ajuta să descoperim ipoteze lipsă
\end{itemize}
\vitem<4-> Uneori afirmațiile evidente nu sunt deloc evidente
\begin{itemize}
\item<4-> E.g., Conjectura lui Kepler, teorema celor 4 culori
\end{itemize}
\vitem<5-> Demonstrațiile constructive pot conduce la metode algoritmice
\end{itemize}
\end{frame}



\begin{subsection}{Inducție structurală pe tipul expresiilor}

\begin{frame}{Determinism puternic}
\begin{theorem}[Limbajul IMP este puternic determinist]
Dacă $\Ss{\c{e,\stare}}{\c{e_1,\stare_1}}$ și $\Ss{\c{e,\stare}}{\c{e_2,\stare_2}}$, atunci $e_1 = e_2$ și $\stare_1 = \stare_2$.
\end{theorem}
\begin{proof}[Idee de demonstrație]
 Fie $P$ proprietatea definită de 

\(P(e) \stackrel{def}{=} \forall s,e_1,s_1,e_2,s_2.\)
\[ \Ss{\c{e,s}}{\c{e_1,s_1}} \wedge \Ss{\c{e,s}}{\c{e_2,s_2}} \implies \c{e_1,s_1} = \c{e_2, s_2}\]

Demonstrăm că $P$ e adevărată pentru toate expresiile IMP prind inducție asupra structurii lui $e$.
\end{proof}
\end{frame}

\begin{frame}{Determinism puternic}
\only<beamer>{
\begin{theorem}[Limbajul IMP este puternic determinist]
Dacă $\Ss{\c{e,\stare}}{\c{e_1,\stare_1}}$ și $\Ss{\c{e,\stare}}{\c{e_2,\stare_2}}$, atunci $e_1 = e_2$ și $\stare_1 = \stare_2$.
\end{theorem}
}

\begin{block}{Definiție (Valoare)}
Spunem că $v$ e valoare, notat $\mathop{valoare}(v)$, dacă $v$ e întreg, boolean, sau $\Sskip$.
Fie $\mathbb{V}$ mulțimea valorilor, adică
\[\mathbb{V} = \mathbb{Z} \cup \{\Strue, \Sfalse\} \cup \{\Sskip\}\]
\end{block}

\begin{block}{Lemă (Valorile nu mai pot fi reduse)}
Dacă $v$ este valoare, atunci pentru orice stare a memoriei $s$, 
\[\c{v,s} \not\rightarrow\]
\end{block}
\end{frame}

\begin{frame}<handout:0>{Definiția sistemului de tranziție pentru limbajul IMP}
$\reg[+]{\Ss{\c{n_1 + n_2,\stare}}{\c{n,\stare}}}{}{n = n_1 + n_2}$
$\reg[$\leq$]{\Ss{\c{n_1 \terminal{<=} n_2,\stare}}{\c{b,\stare}}}{}{b = (n_1 \leq n_2)}$

\vfill { $\reg[OpS]{\Ss{\c{e_1 \mathrel{o} e_2,\stare}}{\c{e_1' \mathrel{o} e_2,\stare'}}}{\Ss{\c{e_1,\stare}}{\c{e_1',\stare'}}}{}$\hfill
$\reg[OpD]{\Ss{\c{n_1 \mathrel{o} e_2,\stare}}{\c{n_1 \mathrel{o} e_2',\stare'}}}{\Ss{\c{e_2,\stare}}{\c{e_2',\stare'}}}{}$}

\vfill $\reg[Loc]{\Ss{\c{\terminal{!} l,\stare}}{\c{n, \stare}}}{}{l\in\Dom(\stare), n = \stare(l)}$
$\reg[Atrib]{\Ss{\c{l \terminal{:=} n,\stare}}{\c{\terminal{skip},\stare[l \mapsto n]}}}{}{l\in\Dom(\stare)}$

\vfill$\reg[AtribD]{\Ss{\c{l \terminal{:=} e ,\stare}}{\c{l \terminal{:=} e' ,\stare'}}}{\Ss{\c{e,\stare}}{\c{e',\stare'}}}{}$

$\reg[Secv]{\Ss{\c{\terminal{skip} \terminal{;} e_2,\stare}}{\c{e_2,\stare}}}{}{}$
$\reg[SecvS]{\Ss{\c{e_1 \terminal{;} e_2},\stare}{\c{e_1' \terminal{;} e_2,\stare'}}}{\Ss{\c{e_1,\stare}}{\c{e_1',\stare'}}}{}$


\vfill $\reg[IfTrue]{\Ss{\c{\Sif  {\terminal{true}} \Sthen e_1 \Selse e_2,\stare}}{\c{e_1,\stare}}}{}{}$ 
$\reg[IfFalse]{\Ss{\c{\Sif {\terminal{false}} \Sthen e_1 \Selse e_2,\stare}}{\c{e_2,\stare}}}{}{}$ 
$\reg[IfS]{
 \Ss{\c{\Sif {e} \Sthen e_1 \Selse e_2,\stare}}{\c{\Sif  {e'} \Sthen e_1 \Selse e_2,\stare'}}
}{
  \Ss{\c{e,\stare}}{\c{e',\stare'}}
}
{}$

\vfill $\reg[While]{\Ss{\c{\Swhile {e_1} \Sdo e_2 \Sdone{},\stare}}{}}{}
{}$

\hfill $\c{\Sif {e_1} \Sthen (e_2\terminal{;} {\Swhile {e_1} \Sdo e_2 \Sdone}) \Selse  {\terminal{skip}},\stare}$

\end{frame}


\begin{frame}{Cazuri de inducție structurală pentru IMP}{}
\begin{itemize}
\item[]\structure{Cazuri de bază.}  
Demonstrăm
\begin{itemize}
\item $P(n)$ pentru orice $n \in \mathbb{Z}$,
\item $P(\Strue)$ și $P(\Sfalse)$
\item $P(\Sskip)$
\item$P(\terminal{!} l)$ pentru orice $l\in \mathbb{L}$
\end{itemize}

\item[] \structure{Recursie simplă.}  
Demonstrăm că dacă $P(e)$ atunci și
\begin{itemize}
\item $P(l \terminal{:=} e)$ pentru orice $l \in \mathbb{L}$
\end{itemize}
\item[] \structure{Recursie dublă.}
Demonstrăm că dacă $P(e_1)$ și $P(e_2)$, atunci și
\begin{itemize}
\item $P(e_1 \terminal{o} e_2)$ pentru orice $o \in \{\leq, +\}$
\item $P(e_1 \terminal{;} e_2)$
\item $P(\Swhile e_1 \Sdo e_2 \Sdone)$ 
\end{itemize}
\item[]\structure{Recursie triplă.}
Demonstrăm că dacă $P(e_1)$, $P(e_2)$ și $P(e_3)$, atunci și 
\begin{itemize}
\item $P(\Sif e_1 \Sthen e_2 \Selse e_3)$ 
\end{itemize}
\end{itemize}
\end{frame}
\end{subsection}


\begin{subsection}{Inducție deductivă pe definiția relației de tip}

\begin{frame}{Proprietatea de a progresa a sistemului de tipuri}
\begin{theorem} Dacă $\tjud{e}{T}$ și $\Dom(\Gamma) \subseteq \Dom(s)$ atunci $e$ este valoare sau există $e'$, $s'$ astfel încât $\Ss{\c{e,s}}{\c{e',s'}}$.
\end{theorem}
\begin{proof}[Idee de demonstrație]
 Fie $P$ proprietatea definită de 

\(P(\tjud{e}{t}) \stackrel{def}{=} \mathop{value}(e) \vee (\forall s.\)
\[ \Dom(\Gamma) \subseteq \Dom(s) \implies \exists e',s'. \Ss{\c{e,s}}{\c{e',s'}}\]

Demonstrăm că $P$ e adevărată pentru toate expresiile care au un tip prin inducție asupra definiției relației de tip.
\end{proof}
\end{frame}


\begin{frame}{Proprietatea de a progresa a sistemului de tipuri}
\only<beamer>{
\begin{theorem} Dacă $\tjud{e}{T}$ și $\Dom(\Gamma) \subseteq \Dom(s)$ atunci $e$ este valoare sau există $e'$, $s'$ astfel încât $\Ss{\c{e,s}}{\c{e',s'}}$.
\end{theorem}}
\begin{block}{Lemă}
Dacă $\tjud{e}{T}$ și $e$ este valoare, atunci 
\begin{description}
\item[Dacă $T = {\Sint}$] atunci există $n$ întreg astfel încât $e = n$
\item[Dacă $T = {\Sbool}$] atunci $e = \Strue$ sau $e = \Sfalse$
\item[Dacă $T = {\Sunit}$] atunci $e = {\Sskip}$
\end{description}
\end{block}
\end{frame}

\begin{frame}{Reguli pentru tipuri}{}

$\reg[tN]{\tjud{n}{\Sint}}{}{n \in \mathbb{Z}}$

\vfill$\reg[tB]{\tjud{b}{\Sbool}}{}{b\in \{\Strue,\Sfalse\}}$

\vfill$\reg[t+]{\tjud{e_1 + e_2}{\Sint}}{\tjud{e_1}{\Sint} \si \tjud{e_2}{\Sint}}{}$
\hfill
$\reg[t$\leq$]{\tjud{e_1 \terminal{<=} e_2}{\Sbool}}{\tjud{e_1}{\Sint} \si \tjud{e_2}{\Sint}}{}$

\vfill$\reg[tIf]{\tjud{\Sif e_1 \Sthen e_2 \Selse e_3}{{T}}}{\tjud{e_1}{\Sbool} \si \tjud{e_2}{{T}} \si \tjud{e_3}{{T}}}{}$

\vfill$\reg[tAtrib]{\tjud{l\terminal{:=}e}{\Sunit}}{\tjud{e}{\Sint}}{\Gamma(l) = \Sintref}$

\vfill$\reg[tLoc]{\tjud{!l}{int}}{}{\Gamma(l) = \Sintref}$

\vfill$\reg[tSkip]{\tjud{\Sskip}{\Sunit}}{}{}$

\vfill$\reg[tSecv]{\tjud{e_1\terminal{;} e_2}{T}}{\tjud{e_1}{\Sunit}\si \tjud{e_2}{T}}{}$

\vfill$\reg[tWhile]{\tjud{\Swhile e_1 \Sdo e_2 \Sdone}{\Sunit}}{\tjud{e_1}{\Sbool} \si \tjud{e_2}{\Sunit}}{}$
\end{frame}


\begin{frame}<handout:0>{Definiția sistemului de tranziție pentru limbajul IMP}
$\reg[+]{\Ss{\c{n_1 + n_2,\stare}}{\c{n,\stare}}}{}{n = n_1 + n_2}$
$\reg[$\leq$]{\Ss{\c{n_1 \terminal{<=} n_2,\stare}}{\c{b,\stare}}}{}{b = (n_1 \leq n_2)}$

\vfill { $\reg[OpS]{\Ss{\c{e_1 \mathrel{o} e_2,\stare}}{\c{e_1' \mathrel{o} e_2,\stare'}}}{\Ss{\c{e_1,\stare}}{\c{e_1',\stare'}}}{}$\hfill
$\reg[OpD]{\Ss{\c{n_1 \mathrel{o} e_2,\stare}}{\c{n_1 \mathrel{o} e_2',\stare'}}}{\Ss{\c{e_2,\stare}}{\c{e_2',\stare'}}}{}$}

\vfill $\reg[Loc]{\Ss{\c{\terminal{!} l,\stare}}{\c{n, \stare}}}{}{l\in\Dom(\stare), n = \stare(l)}$
$\reg[Atrib]{\Ss{\c{l \terminal{:=} n,\stare}}{\c{\terminal{skip},\stare[l \mapsto n]}}}{}{l\in\Dom(\stare)}$

\vfill$\reg[AtribD]{\Ss{\c{l \terminal{:=} e ,\stare}}{\c{l \terminal{:=} e' ,\stare'}}}{\Ss{\c{e,\stare}}{\c{e',\stare'}}}{}$

$\reg[Secv]{\Ss{\c{\terminal{skip} \terminal{;} e_2,\stare}}{\c{e_2,\stare}}}{}{}$
$\reg[SecvS]{\Ss{\c{e_1 \terminal{;} e_2},\stare}{\c{e_1' \terminal{;} e_2,\stare'}}}{\Ss{\c{e_1,\stare}}{\c{e_1',\stare'}}}{}$


\vfill $\reg[IfTrue]{\Ss{\c{\Sif  {\terminal{true}} \Sthen e_1 \Selse e_2,\stare}}{\c{e_1,\stare}}}{}{}$ 
$\reg[IfFalse]{\Ss{\c{\Sif {\terminal{false}} \Sthen e_1 \Selse e_2,\stare}}{\c{e_2,\stare}}}{}{}$ 
$\reg[IfS]{
 \Ss{\c{\Sif {e} \Sthen e_1 \Selse e_2,\stare}}{\c{\Sif  {e'} \Sthen e_1 \Selse e_2,\stare'}}
}{
  \Ss{\c{e,\stare}}{\c{e',\stare'}}
}
{}$

\vfill $\reg[While]{\Ss{\c{\Swhile {e_1} \Sdo e_2 \Sdone{},\stare}}{}}{}
{}$

\hfill $\c{\Sif {e_1} \Sthen (e_2\terminal{;} {\Swhile {e_1} \Sdo e_2 \Sdone}) \Selse  {\terminal{skip}},\stare}$

\end{frame}

\end{subsection}

\begin{subsection}{Inducție deductivă pe definiția sistemului de tranziție}

\begin{frame}{Proprietatea de conservare a tipului}

\begin{theorem}
Dacă $\tjud{e}{T}$, $\Dom(\Gamma) \subseteq \Dom(s)$ și $\Ss{\c{e,s}}{\c{e',s'}}$, \\atunci 
$\tjud{e'}{T}$ și $\Dom(\Gamma) \subseteq \Dom(s')$.
\end{theorem}

\begin{proof}[Idee de demonstrație]
 Fie $P$ proprietatea definită de 

\(P(\Ss{\c{e,s}}{\c{e',s'}}) \stackrel{def}{=} \forall \Gamma,T.\)
\[\tjud{e}{T} \wedge \Dom(\Gamma)\subseteq\Dom(s) \implies \tjud{e'}{T} \wedge \Dom(\Gamma)\subseteq\Dom(s') 
\]

Demonstrăm că $P$ e adevărată pentru întreg sistemul de tranziție asociat semanticii IMP prin inducție asupra definiției sistemului de tranziție.
\end{proof}

\end{frame}

\begin{frame}{Definiția sistemului de tranziție pentru limbajul IMP}
$\reg[+]{\Ss{\c{n_1 + n_2,\stare}}{\c{n,\stare}}}{}{n = n_1 + n_2}$
$\reg[$\leq$]{\Ss{\c{n_1 \terminal{<=} n_2,\stare}}{\c{b,\stare}}}{}{b = (n_1 \leq n_2)}$

\vfill { $\reg[OpS]{\Ss{\c{e_1 \mathrel{o} e_2,\stare}}{\c{e_1' \mathrel{o} e_2,\stare'}}}{\Ss{\c{e_1,\stare}}{\c{e_1',\stare'}}}{}$\hfill
$\reg[OpD]{\Ss{\c{n_1 \mathrel{o} e_2,\stare}}{\c{n_1 \mathrel{o} e_2',\stare'}}}{\Ss{\c{e_2,\stare}}{\c{e_2',\stare'}}}{}$}

\vfill $\reg[Loc]{\Ss{\c{\terminal{!} l,\stare}}{\c{n, \stare}}}{}{l\in\Dom(\stare), n = \stare(l)}$
$\reg[Atrib]{\Ss{\c{l \terminal{:=} n,\stare}}{\c{\terminal{skip},\stare[l \mapsto n]}}}{}{l\in\Dom(\stare)}$

\vfill$\reg[AtribD]{\Ss{\c{l \terminal{:=} e ,\stare}}{\c{l \terminal{:=} e' ,\stare'}}}{\Ss{\c{e,\stare}}{\c{e',\stare'}}}{}$

$\reg[Secv]{\Ss{\c{\terminal{skip} \terminal{;} e_2,\stare}}{\c{e_2,\stare}}}{}{}$
$\reg[SecvS]{\Ss{\c{e_1 \terminal{;} e_2},\stare}{\c{e_1' \terminal{;} e_2,\stare'}}}{\Ss{\c{e_1,\stare}}{\c{e_1',\stare'}}}{}$


\vfill $\reg[IfTrue]{\Ss{\c{\Sif  {\terminal{true}} \Sthen e_1 \Selse e_2,\stare}}{\c{e_1,\stare}}}{}{}$ 
$\reg[IfFalse]{\Ss{\c{\Sif {\terminal{false}} \Sthen e_1 \Selse e_2,\stare}}{\c{e_2,\stare}}}{}{}$ 
$\reg[IfS]{
 \Ss{\c{\Sif {e} \Sthen e_1 \Selse e_2,\stare}}{\c{\Sif  {e'} \Sthen e_1 \Selse e_2,\stare'}}
}{
  \Ss{\c{e,\stare}}{\c{e',\stare'}}
}
{}$

\vfill $\reg[While]{\Ss{\c{\Swhile {e_1} \Sdo e_2 \Sdone{},\stare}}{}}{}
{}$

\hfill $\c{\Sif {e_1} \Sthen (e_2\terminal{;} {\Swhile {e_1} \Sdo e_2 \Sdone}) \Selse  {\terminal{skip}},\stare}$

\end{frame}

\end{subsection}


\begin{frame}{Proprietatea de siguranță a sistemului de tipuri}
{programele bine formate nu se împotmolesc}
\begin{theorem}
Dacă $\tjud{e}{T}$, $\Dom(\Gamma) \subseteq \Dom(s)$ și ${\c{e,s}}\longrightarrow^\ast{\c{e',s'}}$, atunci $e'$ este valoare sau există $e''$, $s''$ astfel încât $\Ss{\c{e',s'}}{\c{e'',s''}}$.
\end{theorem}


\begin{proof}[Idee de demonstrație]
 Fie $P$ proprietatea definită de 
\(P(k) \stackrel{def}{=} \forall \Gamma,e,T,s,e',s'. \)
\(\tjud{e}{T} \wedge \Dom(\Gamma)\subseteq\Dom(s) \wedge {\c{e,s}}\longrightarrow^k{\c{e',s'}} \implies \tjud{e'}{T} \wedge \Dom(\Gamma)\subseteq\Dom(s')\)
\begin{itemize}
\item Demonstrăm că $P$ e adevărată pentru orice număr natural $k$ prin inducție asupra lui $k$ 
\begin{itemize}
\item Folosim Prop. de conservare pentru pasul de inducție.
\end{itemize}
\item Aplicăm Prop. de progres lui $\c{e',s'}$ din ipoteza teoremei.
\end{itemize}
\end{proof}

\end{frame}

\begin{frame}{Tehnici de demonstrație}{Rezumat}
\begin{description}
\item[Determinism] Inducție structurală pe definiția expresiilor \structure{e}
\item[Progres] Inducție deductivă pe definiția relației de tip \structure{$\tjud{e}{T}$}
\item[Conservarea tipurilor] Inducție deductivă pe definiția sistemului de tranziție \structure{$\Ss{\c{e',s'}}{\c{e'',s''}}$}
\item[Siguranță] Inducție matematică pe lungimea secvenței de tranziții \structure{$\longrightarrow^k$}
\end{description}

\begin{block}{Cum alegem pe ce facem inducție?}
Cu grijă... în general alegem metoda cea mai la îndemână, încercăm să rescriem proprietatea pentru a vedea ce se cuantifică universal.
\end{block}
\end{frame}

%
%%\begin{frame}{Inducție deductivă}{Exemplu}
%%\begin{theorem}[Corespondență între semantica naturală și mașina SMC]
%%Dacă $\S{\c{e,s}}{n}$ atunci $\c{e\cdot c, r, s} \longrightarrow^\ast \c{c, n\cdot r, s}$.
%%\end{theorem}
%%\begin{block}{Demonstrație (inducție deductivă asupra relației $\S{}{}$)}
%%  \begin{description} 
%%     \item[$\reg{\S{\c{n,s}}{n}}{}{n \mbox{ întreg}}$:] $\c{n \cdot c, r, s} \xrightarrow{\mbox{IntCt}} \c{c, n\cdot r, s}$ 
%%    \item[$\reg{\S{\c{{\mathrel{!} l},s}}{n}}{}{l \mbox{ locație, cu $l\in Dom(s)$ și $s(l)=n$}}$ :] 
%%$\c{{\mathrel{!} l} \cdot c, r, s} \xrightarrow{\mbox{Loc}} \c{c, n\cdot r, s}$ 
%%    \item[$\reg{\S{\c{e_1\mathrel{o} e_2,s}}{n}}{\S{\c{e_1,s}}{n_1}\si \S{\c{e_2,s}}{n_2}}{o \mbox{ operație întreagă și $n = n_1 \mathrel{o} n_2$}}$ :]
%%
%%     \item[]\vspace{-1ex}$\begin{array}{l}
%%  \c{e_1\mathrel{o} e_2 \cdot c, r, s} \xrightarrow{\mbox{IntOpC}} \c{e_1 \cdot e_2 \cdot o \cdot c, r, s} \xrightarrow{\mbox{ip. ind}}
%%\\
%%  \c{e_2 \cdot o \cdot c, n_1 \cdot r, s}  \xrightarrow{\mbox{ip. ind}}   \c{o \cdot c, n_2 \cdot n_1 \cdot r, s}  \xrightarrow{\mbox{IntOp}}
%%\\
%%   \c{c, n \cdot r, s}
%%\end{array} $
%%  \end{description}
%%\end{block}
%%\end{frame}
\end{section}

\end{document}



