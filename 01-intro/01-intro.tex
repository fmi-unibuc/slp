\documentclass[xcolor=pdftex,romanian,colorlinks]{beamer}

\usepackage{../tslides}
\usepackage[all]{xy}

\title[SLP---Introducere]{Semantica limbajelor de programare}
\subtitle{(bazat pe cursul omonim de la Universitatea Cambridge)}
\begin{document}
\begin{frame}
  \titlepage
\end{frame}

\begin{section}{Motivație}
\begin{frame}{Ce înseamnă semantica formală?}%
{Ce definește un limbaj de programare?}
\begin{description}
\vitem[Sintaxa] Simboluri de operație, cuvinte cheie, descriere (formală) a programelor/expresiilor bine formate
\vitem[Practica] Un limbaj e definit de modul cum poate fi folosit
\begin{itemize}
\item Manual de utilizare și exemple de bune practici
\item Implementare (compilator/interpretor)
\item Instrumente ajutătoare (analizor de sintaxă, depanator)
\end{itemize}  
\vitem[Semantica?] Ce înseamnă / care e comportamentul unei instrucțiuni?
\begin{itemize}
\item De cele mai multe ori se dă din umeri și se spune că \structure{Practica} e suficientă 
\end{itemize}
\end{description}
\end{frame}

\begin{frame}{La ce folosește semantica}
\begin{itemize}
\item Să înțelegem un limbaj în profunzime
\begin{itemize}
\item Ca programator: pe ce mă pot baza când programez în limbajul dat
\item Ca implementator al limbajului: ce garanții trebuie să ofer
\end{itemize}
\vitem Ca instrument în proiectarea unui nou limbaj / a unei extensii
 \begin{itemize}
\item Înțelegerea componentelor și a relațiilor dintre ele
\item Exprimarea (și motivarea)  deciziilor de proiectare
\item Demonstrarea unor proprietăți generice ale limbajului\\
E.g., execuția nu se va bloca pentru programe care trec de analiza tipurilor
\end{itemize}
\vitem Ca bază pentru demonstrarea corectitudinii programelor.
\end{itemize}
\end{frame}

\begin{frame}{Caracteristici care \underline{nu} sunt principale pentru cursul acesta}
\begin{block}{În acest curs \alert{nu} vom insista pe:}
\begin{itemize}
\item Proiectarea și implementarea limbajelor de programare
\item Trăsături avansate ale limbajelor de programare
\item \ldots
\end{itemize}
\end{block}

\begin{block}{De ce?}
\begin{itemize}
\item Pentru ca ne lipsesc elementele/metodele de bază
\item Acest curs stabilește cadrul pentru dezvoltări ulterioare
\end{itemize}
\end{block}

\begin{block}{Când?}
\begin{itemize}
\item (Potențial) alte cursuri la master
\item La doctorat 
\end{itemize}
\end{block}
\end{frame}

\begin{frame}[fragile]{C}
\onslide<2>
\begin{block}{}
\begin{verbatim}
  int main(void) {
    int x = 3;
    return x+++x+++x+++x++;
  }
\end{verbatim}
\end{block}
\end{frame}

\begin{frame}[fragile]{C}

\begin{block}{}
\begin{verbatim}
  int main(void) {
    int x = 0;
    return (x = 1) + (x = 2);
  }
\end{verbatim}
\end{block}
\onslide<2>
Conform standardului C, comportamentul programului este \structure{nedefinit}.
\begin{itemize}
\item GCC4, MSVC: valoarea întoarsă e \alert{4}
\item GCC3, ICC, Clang: valoarea întoarsă e \alert{3}
\end{itemize}
\end{frame}

\begin{frame}[fragile]{C}

\begin{block}{}
\begin{verbatim}
  int r; 
  int f(int x) {
    return (r = x);
  }
  int main() {
    return f(1) + f(2), r;
  }
\end{verbatim}
\end{block}

\onslide<2> Conform standardului C, comportamentul programului este \structure{subspecificat}:\\
poate întoarce atât valoarea \alert{1} cât și \alert{2}.
\end{frame}

\begin{frame}[fragile]{C\#}
\onslide<2>
\begin{verbatim}
delegate int IntThunk();
class M {
    public static void Main() {
        IntThunk[] funcs = new IntThunk[11];
        for (int i = 0; i <= 10; i++) {
            funcs[i] = delegate() { return i; };
        }
        foreach (IntThunk f in funcs) {
            System.Console.WriteLine(f());
        }
    }
}
\end{verbatim}
\end{frame}


\begin{frame}[fragile]{C++}
\onslide<2>
\begin{verbatim}
#include <iostream>
#include <array>
#include <functional>
using namespace std;

int main() {
    array<function<int()>,11> funcs;
    for (auto i = 0; i < 11; i++) {
    	funcs[i] = [&]() {return i;};
    }
    for (auto f : funcs) {
    	cout << f() << endl;
    }
}
\end{verbatim}
\end{frame}

\begin{frame}[fragile]{JavaScript}
\onslide<2>
\begin{verbatim}
function bar(x) { 
  return function() { var x = 5; return x; }; 
}
var f = bar(200);
f()
\end{verbatim}
\end{frame}

\begin{frame}[fragile]{JavaScript}
\begin{verbatim}
function bar(x) { 
  return function() { var x = x; return x; }; 
}
var f = bar(200);
f()
\end{verbatim}
\end{frame}

\begin{frame}[fragile]{Java}
\onslide<2>
\small{\begin{verbatim}
class Main {
    interface F<A, B> {          B a(A x);         }
    static <A, B, C> F<A, C> c(final F<A, B> f, final F<B, C> g) {
        return new F<A, C>() { 
            public C a(A x) {  return g.a(f.a(x)); }
        };
    }
    public static void main(String[] args) {
        final Integer a = 2, b = 1;
        F<Integer, Integer> f = new F<Integer, Integer>() { 
            public Integer a(Integer x) { return x + b; } };
        F<Integer, Integer> g = new F<Integer, Integer>() { 
            public Integer a(Integer x) { return a * x; } };
        F<Integer, Integer> h = c( f, g );
        System.out.println(h.a(5));
    }
}
\end{verbatim}}
\end{frame}

\begin{frame}{De ce semantică formală?}
\begin{itemize}
\vitem Avantaje la implementare
\begin{itemize}
\item Oferă o descriere a comportamentului independent de implementare
\item Oferă o modalitate de a justifica corectitudinea  optimizărilor.
\end{itemize} 
\vitem Verificarea programelor
\begin{itemize}
\item Oferă o bază pentru specificarea și verificarea corectitudinii programelor.
\end{itemize}
\vitem Validarea proiectării limbajelor
\begin{itemize}
\item O definiție formală poate releva potențiale interacțiuni nedorite între elementele limbajului.
\item Formalizarea poate duce la o restructurare mai riguroasă a limbajului
\end{itemize}
\end{itemize}
\end{frame}

\begin{frame}{Privire de ansamblu}{}
\xymatrix@C=1em@R=5em{
\mbox{Limbaje formale} \ar[dr]& \mbox{Logică}\ar[d] & \mbox{C/Java/\ldots}\ar[dl] & \mbox{Calculabilitate}\ar[dll]\\
&\mbox{\large \color{blue}Semantica}\ar[dl]\ar[d]\ar[dr]\ar[drr] \\
\mbox{Compilare} & \mbox{Verificare} & \mbox{Concurență} & \mbox{Semantica denotațională} 
}
\end{frame}



\begin{frame}{Tipuri de semantică}
\begin{itemize}
\vitem Operațională
\begin{itemize}
\item Descrie execuția programului
\item Asociază reguli de deducție elementelor de limbaj
\item Pașii de execuție devin pași de demonstrație
\end{itemize}
\vitem Denotațională
\begin{itemize}
\item Asociază elementelor de limbaj obiecte matematice abstracte bine specificate dintr-un univers matematic (domeniu) ales în mod convenabil
\end{itemize}
\vitem  Axiomatică
\begin{itemize}
\item Asociază elementelor de limbaj contracte între proprietățile mediului înaintea execuției și cele de după execuție
\end{itemize}
\end{itemize}
\end{frame}

\end{section}

\begin{section}{Descriere curs}
\begin{frame}{Structură cursuri}{}
\begin{itemize}
\vitem<1-> Semantică operațională 
\only<1>{%
\begin{itemize}
\item Mașini abstracte
\item Semantica evaluării
\item Semantică tranzițională
\end{itemize}%
}
\vitem<2-> Tehnici de demonstrație
\only<2>{%
\begin{itemize}
\item Inducție structurală
\item Reguli de deducție
\item Definiții inductive
\item Inducție deductivă
\end{itemize}%
}
\vitem<3-> Echivalență semantică
\vitem<4-> Elemente de programare funcțională 
\only<4>{%
\begin{itemize}
\item Evaluare prin substituție
\item Apel prin valoare/apel prin nume
\item Sisteme de tipuri
\item Recursie locală
\end{itemize}%
}
\vitem<5-6> Programe interactive
\only<5>{%
\begin{itemize}
\item Intrare/Ieșire
\item Bisimilaritate ca echivalență de programe interactive
\item Procese și comunicare
\end{itemize}%
}
\end{itemize}
\end{frame}

\begin{frame}{Obiective}
\begin{itemize}
\vitem Formalizarea și înțelegerea unor concepte de bază din semantica (operațională a) limbajelor de programare
\vitem Deprinderea abilității de a formaliza
\begin{itemize}
\item Definiția unui limbaj de programare
\item Proprietăți ale limbajului și/sau ale programelor
\item Argumente privind validitatea programelor 
\item Argumente privind consistența și adecvarea definițiilor
\end{itemize}
\vitem Deprinderea abilitățiii de a înțelege articole de cercetare legate de semantica limbajelor de programare
\end{itemize}
\end{frame}

\begin{frame}{Resurse}
\begin{itemize}
\item Pagina Moodle a cursului \url{http://moodle.fmi.unibuc.ro/course/view.php?id=522} 
\begin{itemize}
\item Prezentările cursurilor, forum
\item Resurse accesibile pe pagina Moodle
\begin{itemize}
\item Cartea lui Winskel --- The Formal Semantics of Programming Languages
\item Notele de curs ale lui Andrew Pitts --- Lecture Notes on Semantics of Programming Languages
\end{itemize}
\end{itemize}
\end{itemize}
\end{frame}
\end{section}


\begin{section}{Evaluare}
\begin{frame}{Evaluare}
\begin{itemize}
\item Examen scris
\begin{itemize}
\item Va consta din probleme
\item Scopul lui e să verifice fixarea cunoștințelor predate
\item Cu acces (limitat) la materiale 
\end{itemize}
\vitem Probleme date la curs / seminar
\begin{itemize}
\item Contează ca puncte la examen
\item Cam la 2 saptămâni o temă valorând 1-2 puncte
\item Temele trebuie predate la seminar în termen de 2 săptămâni
\end{itemize}
\end{itemize}
\end{frame}

\begin{frame}{Colaborare la teme}
\begin{itemize}
\item Sunteți încurajați să discutați modalitațile de a rezolva o problemă
\item Este mai puțin încurajată folosirea codului scris de altcineva / scrierea împreună a codului
\item În caz că folosiți cod care nu l-ați conceput, marcați acest lucru clar în sursă
\item Vom folosi soft anti-plagiat pentru a verifica tentative de plagiere.
\end{itemize}
\end{frame}

\end{section}

\end{document}



