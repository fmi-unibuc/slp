\documentclass[xcolor=pdftex,romanian,colorlinks]{beamer}

\usepackage{../tslides}

\title[SLP---Introducere]{Semantica Limbajelor de Programare}
\subtitle{Introducere}
\begin{document}
\begin{frame}
  \titlepage
\end{frame}

\begin{section}{Motivație}
  \begin{frame}{Ce înseamnă semantica formală?}%
  {Ce definește un limbaj de programare?}
  \begin{description}
  \vitem[Sintaxa] Simboluri de operație, cuvinte cheie, descriere (formală) a programelor/expresiilor bine formate
  \vitem[Practica] Un limbaj e definit de modul cum poate fi folosit
  \begin{itemize}
  \item Manual de utilizare și exemple de bune practici
  \item Implementare (compilator/interpretor)
  \item Instrumente ajutătoare (analizor de sintaxă, depanator)
  \end{itemize}
  \vitem[Semantica?] Ce înseamnă / care e comportamentul unei instrucțiuni?
  \begin{itemize}
  \item De cele mai multe ori se dă din umeri și se spune că \structure{Practica} e suficientă
  \end{itemize}
  \end{description}
  \end{frame}

  \begin{frame}{La ce folosește semantica}
  \begin{itemize}
  \item Să înțelegem un limbaj în profunzime
  \begin{itemize}
  \item Ca programator: pe ce mă pot baza când programez în limbajul dat
  \item Ca implementator al limbajului: ce garanții trebuie să ofer
  \end{itemize}
  \vitem Ca instrument în proiectarea unui nou limbaj / a unei extensii
   \begin{itemize}
  \item Înțelegerea componentelor și a relațiilor dintre ele
  \item Exprimarea (și motivarea)  deciziilor de proiectare
  \item Demonstrarea unor proprietăți generice ale limbajului\\
  E.g., execuția nu se va bloca pentru programe care trec de analiza tipurilor
  \end{itemize}
  \vitem Ca bază pentru demonstrarea corectitudinii programelor.
  \end{itemize}
  \end{frame}
\end{section}

\begin{frame}{Organizare și evaluare}
\begin{itemize}
\vitem Structura cursului: 2C + 1L
\vitem Evaluare: Nota finala = 1 + Examen + Teme
\vfill
\structure{Observații:}
\begin{enumerate}
\item Promovare: nota finala >=5

\item Examenul poate fi înlocuit de un proiect
\end{enumerate}
\end{itemize}
\end{frame}

\begin{section}{Evaluare}
\begin{frame}{Teme de laborator}
\begin{itemize}
\vitem Noțiunile discutate la curs vor fi însoțite de formalizări pe calculator care vor trebui modificate / extinse / regândite ca parte din temă.
\vitem \alert{Nu subestimați timpul de lucru}. Teme odată la 2 săptămâni, o temă poate conține mai multe probleme.
\vitem De la lansarea temei (pe Moodle), veți avea cam 2 săptămâni pentru a rezolva problemele.
\vitem Soluțiile vor fi trimise pe Moodle și, dacă e cazul, prezentate la laborator.
\end{itemize}
\end{frame}

\begin{frame}{Examenul scris}
\begin{itemize}
\item Va consta din probleme
\vitem Scopul lui e să verifice fixarea cunoștințelor predate
\vitem Cu acces (limitat) la materiale tipărite
\vitem E de așteptat ca nota de la examen să fie în concordanță cu cea de la laborator
\begin{itemize}
\item În caz contrar se vor reevalua amândouă, trebuind să dovediți oral că munca vă aparține
\end{itemize}
\end{itemize}
\end{frame}

\begin{frame}{Proiect---Opțional}{Doar pentru cei cu interes deosebit în limbaje de programare}
\begin{itemize}
\item Definirea nucleului unui limbaj (nou) de programare; sau
\vitem Extinderea unei definiții existente într-un mod interesant și netrivial
\vitem Stabilirea temei: de comun acord, \structure{înainte de jumătatea semestrului}
\vitem \structure{Termen de predare:} înainte de examenul final
\vitem Posibilitate de transformare în proiect pentru lucrarea de dizertație
\end{itemize}
\end{frame}

\begin{frame}{Colaborare}

\begin{block}{Activități încurajate}
\begin{itemize}
\item Discuții despre probleme și modalități generale de a le aborda
\item Folosirea forumului pentru a cere clarificări pentru folosul tuturor
\end{itemize}
\end{block}

\vfill\begin{alertblock}{Activități intolerate}
\begin{itemize}
\item Scrierea împreună a codului
\item Copierea soluțiilor de la alți colegi
\item Consecințe:
\begin{enumerate}
\item Anularea temei / examenului (la prima abatere)
\item Nota 1 la curs și referat în vederea exmatriculării!
\end{enumerate}
\end{itemize}
\end{alertblock}
\end{frame}
\end{section}

\begin{section}{Descriere curs}
\begin{frame}{Listă de subiecte}{În funcție de timp/interes}
\begin{itemize}
\item Paradigme standard de semantică operațională
\begin{itemize}
\item Semantica evaluării (big-step)
\item Semantica tranzițională (small-step)
\end{itemize}
\vitem Platforma \K de definire a limbajelor de programare
\vitem Tipuri de limbaje de programare
\begin{itemize}
\item Funcționale, Imperative, Orientate obiect, Logice
\end{itemize}
\vitem Tipuri de trăsături de limbaj
\begin{itemize}
\item excepții, concurență, comunicare, sincronizare
\end{itemize}
\end{itemize}
\end{frame}

\begin{frame}{Obiective}
\begin{itemize}
\item Formalizarea și înțelegerea conceptelor de bază în proiectarea limbajelor de programare
\vitem Specificarea și definirea (într-un mod declarativ) a limbajelor
\begin{itemize} \item nu implementarea lor
\end{itemize}
\vitem Deprinderea facilității de a defini limbaje folosind \K
\begin{description}
\item[Avantaje] Modular, dinamic, elegant, minim de dependență între diferitele caracteristici ale limbajelor
\item[Bonus] interpretor și cadru de analiză a execuțiilor
\end{description}
\end{itemize}
\end{frame}

\begin{frame}{Resurse}
\begin{itemize}
\item Pagina Moodle a cursului
 \url{http://moodle.fmi.unibuc.ro/course/view.php?id=522}
\begin{itemize}
\item Prezentările cursurilor, forum, teme
\end{itemize}
\vitem \url{http://k-framework.org}
\begin{itemize}
\item Tutoriale, example, interfață web pentru editare/modificare/execuție de definiții și programe
\end{itemize}
\end{itemize}
\end{frame}
\end{section}



\end{document}



