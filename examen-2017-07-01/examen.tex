\documentclass[addpoints,12pt,a4paper,answers]{exam}
%
\usepackage[none]{hyphenat} \usepackage{fullpage}
\usepackage{../tdefinition}
%\usepackage{fancyhdr}
%\pagestyle{fancy}
%\lhead{}
%}
%\rhead{Page \thepage}

\newcommand{\bs}{\char`\\} \newcommand{\bt}{\char`\`}
\newcommand{\us}{\char`\_}
\pagestyle{empty}


\newcommand{\getHeight}[1]{%
%\newdimen\height
%\setbox0=\vbox{#1}
%\height=\ht0 \advance\height by \dp0
%\vspace*{1.1\height}
%#1%
}

\newenvironment{solutie}{\par\hspace*{-9em}\begin{minipage}{.98\paperwidth}
\hrulefill {\bf Rezolvare} \hrulefill}{\hrulefill\end{minipage}}

\begin{document}

\begin{center}

%\makebox[\textwidth]{Prenumele, numele și grupa: \enspace\hrulefill}
%\vspace{0.1in}

Semantica limbajelor de programare---Examen\footnote{1 punct din oficiu} \hfill  1 iulie 2017 \\ \ \\

\end{center}

\pointpoints{punct}{puncte}

\begin{questions}
\question[3] \textbf{Execuții. }
Să se descrie formal execuția programului IMP
\begin{asciiml}
if !x <= 7 then m := !x else m := !y
\end{asciiml}

\vspace{-1ex}din starea inițială \(\{x \mapsto 0, y\mapsto 1\}\)
\begin{parts}
\part[1] Folosind mașina abstractă SMC 
\part[1] Folosind semantica evaluării.
%\newpage

\part[1] Folosind semantica tranzițională.
\end{parts}

\question[1] \textbf{Echivalență de programe. }
Arătați că, pentru orice expresii \lstinline$c, et, ef, e$,
programele următoare sunt semantic echivalente
\begin{asciiml}
if c then (et ; e) else (ef ; e)

(if c then et else ef) ; e
\end{asciiml}

\vspace{-1ex}\question[1] \textbf{Operatori de legare. } Subliniați aparițiile libere ale variabilelor / indicați legătura fiecărei apariții legate a unei variabile printr-o săgeată către parametrul care o leagă.

\begin{asciiml}
fun (x) -> (fun (y) -> let x = (let rec x = x + y in x x) + x in x + z) (fun z -> x y z)
\end{asciiml}

\vspace{-1ex}\question[3] \textbf{Siguranța tipurilor. }
Extindem sintaxa limbajului $\Sfun$-IMP cu:
\vspace{-1ex}\begin{syntaxBlock}{\nonTerminal{e}}
\syntax{e \terminal{when} e \terminal{otherwise} e}{}
\end{syntaxBlock}

\vspace{-4ex}Semantica tranzițională este dată de următoarele reguli:

$\reg[SwhenS]{\Ss{\c{e \terminal{when} c \terminal{otherwise} o,s}}{\c{e' \terminal{when} c \terminal{otherwise} o,s'}}}{\Ss{\c{e,s}}{\c{e',s'}}}{}$
și

$\reg[Swhen]{\Ss{\c{v \terminal{when} c \terminal{otherwise} o,s}}{\c{\terminal{if} c \terminal{} v \terminal{then} v \terminal{else} o,s}}}{}{}$

\begin{parts}
\part[1] Scrieți regula de tipuri corespunzătoare
\part[1] Demonstrați proprietatea de progres
\part[1] Demonstrați proprietatea de conservare a tipurilor
\end{parts}

\question[3\half] \textbf{Inferența tipurilor. }
Se dă expresia $e$:
\begin{asciiml}
let a = fun (b) -> fun (c) -> fun(d) ->
          let rec e = fun (a) -> if b a then c else d a (e (a - 1))
          in  e
in a
\end{asciiml}
\begin{parts}
\vspace{-1ex}\part[\half] Descrieți în cuvinte rezultatul evaluării expresiei $e$
\part[1] Găsiți un tip $\tau$ astfel încât $\emptyset \vdash e : \tau$
\part[2] Demonstrați că există un tip $\tau$ astfel încât $\emptyset \vdash e: \tau$
\end{parts}


\question[2] \textbf{Subtipuri. }
Se dau expresia $e: \{a=0, b=(x\, y).a\}$ și mediul de tipuri
\[\begin{array}{lcl}\Gamma&=& x: (\{a: \textit{int}\} \rightarrow \{a:\{\}, b:\textit{int}\})\rightarrow \{x:\textit{int}, a: \textit{bool}\},\\ && y: \{\} \rightarrow \{b: \textit{int}, a:\{x: \textit{bool}\}\}.\end{array}\]

\begin{parts}
\part[\half] Găsiți un tip $\tau$ astfel încât $\Gamma \vdash e : \tau$. 

\part[1\half] Demonstrați că există un tip $\tau$ astfel încât $\Gamma \vdash e : \tau$. 
\end{parts}

\end{questions}


%Ciornă
\end{document}
