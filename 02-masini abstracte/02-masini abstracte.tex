\documentclass[handout,xcolor=pdftex,romanian,colorlinks]{beamer}
\usepackage{../tslides}

\title[SLP---Mașini Abstracte]{Semantica Limbajelor de Programare}
\subtitle{Sisteme de tranziție, Sintaxă abstractă, mașină de execuție abstractă}
\begin{document}
\maketitle

%!TEX root = 02-masini abstracte.tex

\begin{section}{IMP}
\begin{frame}[fragile]{Limbajul IMP}
IMP este un limbaj \structure{IMP}erativ foarte simplu.
\begin{block}{Ce conține}
\begin{minipage}{.49\columnwidth}
\begin{itemize}
\vitem Expresii
\begin{itemize}
  \item Aritmetice%\hfill $x + 3$
  \item Booleene%\hfill $\Snot(x > 7)$
\end{itemize}
\vitem Comenzi
\begin{itemize}
  \item De atribuire%\hfill $x = 5;$ 
  \item Condiționale%\hfill $\Sif(x > 7)\; \{x =5; \} \Selse \{x = 0;\}$
  \item De ciclare  %\hfill$\Swhile (x > 7)\; \{x = x - 1;\}$
\end{itemize}
\end{itemize}
\end{minipage}
\begin{minipage}{.49\columnwidth}
\begin{asciic}
 x := 10 ;
(sum := 0 ;
 while (0 <= !x)  do 
   (sum := !sum + !x ;
    x := !x - 1
   )
)
\end{asciic}
\end{minipage}
\end{block}
\begin{block}{Ce nu conține}
\begin{itemize}
\item Expresii cu efecte laterale
\item Proceduri și funcții
\item Schimbări abrupte de control
\end{itemize}
\end{block}
\end{frame}

 \begin{frame}{Sintaxă formală}{Backus Naur Form}
  \begin{itemize}
   \item Pentru gramatici (generative) independente de context
   \item Producții
       --- generează termeni prin expandare (rescriere)

\alert{\renewcommand{\syntaxKeyword}{}\syntax[\Stmt]{\color{black}\Sif\;\BExp\Sthen\Block\Selse\Block}{}
	\syntaxCont[\Stmt]{\color{black}\Id \terminal{:=}\AExp\terminal{;}}{}}
   \item Categorii sintactice (neterminale) 
   \begin{itemize}
    \item descriu tipurile de sintaxă
     \begin{itemize}
      \item Tipuri lexicale: 
		întregi (\structure{n}), booleeni (\structure{b}), locații (\structure{l}),\\
		simboluri de operații aritmetice (\structure{iop}) sau booleene (\structure{bop}).
	  \item Tipuri construite (cu majusculă):
     %$\AExp$ (expr. aritmetice), $\BExp$ (expr. Booleene), \\$\Stmt$ (instrucțiuni), $\Block$ (bloc de instrucțiuni), $\Pgm$ (program)
     \end{itemize}
   \end{itemize}
   \item Cuvinte cheie (terminale, încep cu literă mică sau simboluri)
   \begin{itemize}
    \item descriu elementele lexicale: 
   $\Sif$, $\Sthen$, $\Selse$,  
   $\Swhile$, $\Sdo$,
   $\terminal{:=}$, $\terminal{!}$, $\terminal{;}$ \ldots
   \end{itemize}
  \end{itemize}
 \end{frame}

\begin{frame}{Sintaxa BNF a limbajului IMP}
\vspace{-5ex}\begin{syntaxBlock}{\AExp}
\alert{
\begin{itemize}
\item[]\renewcommand{\syntaxKeyword}{}
\syntax{\Int\Smid\terminal{!}\Id \Smid \AExp\iop \AExp}{}
\item[]\renewcommand{\defSort}{\BExp}
\syntax{\Bool \Smid \AExp\bop \AExp}{}
%\item[]\renewcommand{\defSort}{\Block}
%\syntax{\terminal{\{}\Stmt\terminal{\}} \Smid \terminal{\{}\terminal{\}}}{} 
\item[]\renewcommand{\defSort}{\Stmt}
\syntax{\Sskip \Smid \Id \terminal{:=} \AExp \Smid  \Stmt\terminal{;}\Stmt}{}
\syntaxCont{\Sif\BExp\Sthen \Block\Selse\Block \Smid \Swhile \BExp\Sdo \Block}{}
\item[]\renewcommand{\defSort}{\Pgm}
\syntax{\AExp \Smid \BExp \Smid \Stmt}{} 
\item[]\renewcommand{\defSort}{\Id}
\syntax{\mbox{locație de memorie / variabilă}}{} 
\\
\renewcommand{\defSort}{\Int}
\syntax{\mbox{număr întreg}}{}
\\
\renewcommand{\defSort}{\Bool}
\syntax{\Strue \mid \Sfalse}{}
\\
\renewcommand{\defSort}{\iop}
\syntax{\terminal{+} \Smid \terminal{-}\Smid \ldots}{}
\renewcommand{\defSort}{\bop}
\syntax{\terminal{=} \Smid \terminal{<}\Smid \ldots}{}
%\syntaxCont{\terminal{and} \Smid \terminal{or}\Smid \ldots}{}
\end{itemize}
}
\end{syntaxBlock}
\end{frame}
\end{section}

\begin{section}{Sisteme de tranziție}
\begin{frame}{Semantica Operațională}
\begin{block}{Scop}
Descrierea formală a execuției unui sistem.
\end{block}
\begin{block}{Idee}
Execuția unui sistem este capturata de evoluția în timp a stării sistemului
\end{block}
\begin{block}{Principii generale}
\begin{itemize}
\item Reprezentăm starea sistemului ca un obiect matematic
\item Reprezentăm evoluția sistemului prin tranziții de la o stare la alta
\end{itemize}
\end{block}
\begin{block}{Formalizare}
\begin{itemize}
\item O execuție posibilă: secvență de tranziții între stările sistemului
\item Spațiul execuțiilor: Sistem de tranziție
\end{itemize}
\end{block}
\end{frame}
\begin{frame}{Sisteme de tranziție}{}
\begin{block}{Definiție (Sistem de tranziție)}
Un sistem de tranziție este dat de 
\begin{itemize}
\item O mulțime {\it Config}
\item O relație „de tranziție“ $\rightarrow \subseteq {\it Config} \times {\it Config}$
\end{itemize}
\end{block}

\begin{itemize}
\item Elementele mulțimii {\it Config} sunt numite \structure{configurații} sau stări
\item Date fiind două configurații cu $(c,c') \in {\it Config}$ scriem $c \rightarrow c'$ și citim $c$ tranziționează/\structure{se transformă în} $c'$ (într-un pas)
\end{itemize}
\end{frame}

\begin{frame}{Definiții derivate}
\begin{description}
\vitem[$\rightarrow^\ast$] $c \rightarrow^\ast c'$ dacă $c$ se transformă în $c'$ in zero, unul, sau mai mulți pași (închiderea reflexiv-tranzitivă a relației $\rightarrow$)
\vitem[$\not\rightarrow$] $c \not\rightarrow$ dacă $c$ nu se poate transforma în nimic altceva.
\vitem[Determinism] Un sistem se numește determinist dacă relația $\rightarrow$ e injectivă:
\[\forall c, c_1, c_2\in {\it Config} . c \rightarrow c_1 \wedge c \rightarrow c_2 \implies c_1 = c_2\]
\end{description}
\end{frame}
\end{section}

\begin{section}{Mașina abstractă SMC}
\begin{frame}{Mașina  \alert{S}tivă--\alert{M}emorie--\alert{C}ontrol  (SMC)}
\begin{block}{Definiție (Configurații)}
O configurație SMC e un triplet $\c{c, r, s}$, definită de:
\begin{itemize}
\item[C] O stivă de \alert{C}ontrol \structure{c} dată de:
\renewcommand{\syntaxKeyword}{}
\syntax[\structure{c}]{\terminal{nil} \Smid \structure{c1} \cdot \structure{c}}{}
\syntax[\structure{c1}]{P \Smid \iop \Smid \bop \Smid \terminal{:=} \Smid \terminal{if} \Smid \terminal{while}}{}
\item[S] O \alert{S}tivă de lucru \structure{r}, conținând rezultate (parțiale) dată de:
\syntax[\structure{r}]{\terminal{nil} \Smid \Pgm \cdot \structure{r} \Smid \Id \cdot \structure{r}}{}
\item[M] o stare a \alert{M}emoriei \structure{m} care asociază valori întregi locațiilor de memorie
\end{itemize}
\end{block}
\end{frame}
\begin{frame}{Starea memoriei}
\begin{block}{}
Starea memoriei unui program IMP la un moment dat este dată de valorile deținute în acel moment de variabilele declarate de program.

\structure{Matematic:} o funcție \alert{parțială} $s : \Id \xrightarrow{\circ}\SInt$ de domeniu finit.
\end{block}
\begin{block}{Notații}
\begin{itemize}
\item Descrierea funcției prin enumerare:
$s = \{ x \mapsto 10, sum \mapsto 0\}$
\item Funcția vidă $\bottom$, nedefinită pentru nici o variabilă
\item Obținerea valorii unei variabile: $s(x)$
\item Suprascrierea valorii unei variabile:

$$s[l\mapsto n] (l') = \left\{\begin{array}{r@{\mbox{, dacă }}l}
s	(l') & l' \neq l \\
n & l' = l
\end{array}
\right.$$
\end{itemize}
\end{block}
\end{frame}

\begin{frame}{Reguli de tranziție SMC}{\only<beamer>{Expresii}}
\begin{block}{Expresii întregi}
\onslide<2->
$\begin{array}{l@{\extracolsep{1em}}r@{{}\longrightarrow{}}l}
\mbox{IntCt}&
\c{n\cdot c,r,s} & \c{c, n \cdot r, s}
\\[1ex]
\mbox{Loc}&
\c{\terminal{!} l\cdot c,r,s} & \c{c, n \cdot r, s}  \Sdaca s(l) = n
\\[1ex]
\mbox{IntOpC}&
\c{(E_1 \iop E_2) \cdot c, r, s} & \c{E_1 \cdot E_2 \cdot \iop \cdot c, r, s}
\\[1ex]
\mbox{IntOp}&
\c{\iop \cdot c, n_2\cdot n_1 \cdot r, s} & \c{c, n \cdot r, s} \Sdaca n_1 \iop n_2 = n
\end{array}$
\end{block}
\begin{block}{Expresii Boole}
\onslide<3>
$\begin{array}{l@{\extracolsep{1em}}r@{{}\longrightarrow{}}l}
\mbox{BoolCt}&
\c{b\cdot c,r,s} & \c{c, b \cdot r, s}
\\[1ex]
\mbox{BoolOpC}&
\c{(E_1 \bop E_2) \cdot c, r, s} & \c{E_1 \cdot E_2 \cdot \bop \cdot c, r, s}
\\[1ex]
\mbox{BoolOp}&
\c{\bop \cdot c, n_2\cdot n_1 \cdot r, s} & \c{c, b \cdot r, s} \Sdaca n_1 \bop n_2 = b
\end{array}$
\end{block}
\end{frame}

\begin{frame}{Reguli de tranziție SMC}{}
\begin{block}{Comenzi}
\onslide<2>
$\begin{array}{lr@{{}\rightarrow{}}l}
\mbox{Skip}&
\c{\Sskip \cdot c,r,s} & \c{c, r, s}
\\[1ex]
\mbox{AtribC}&
\c{l \terminal{:=} E \cdot c,r,s} & \c{E \cdot \terminal{:=} \cdot c, l \cdot r, s} 
\\[1ex]
\mbox{Atrib}&
\c{\terminal{:=}\cdot c,n \cdot l\cdot r,s} & \c{c, r, s[l\mapsto n]} 
\\[1ex]
\mbox{CondC}&
\c{(\Sif B \Sthen C_1 \Selse C_2) \cdot c, r, s} & \c{B \cdot \Sif \cdot c, C_1 \cdot C_2 \cdot r, s}
\\[1ex]
\mbox{CondT}&
\c{\Sif \cdot c, \Strue \cdot C_1 \cdot C_2 \cdot r, s} & \c{C_1 \cdot c, r, s}
\\[1ex]
\mbox{CondF}&
\c{\Sif \cdot c, \Sfalse \cdot C_1 \cdot C_2 \cdot r, s} & \c{C_2 \cdot c, r, s}
\\[1ex]
\mbox{Secv}&
\c{(C_1 \terminal{;} C_2) \cdot c, r, s} & \c{C_1 \cdot C_2 \cdot c, r, s}
\end{array}$
\end{block}
\end{frame}

\begin{frame}{Reguli de tranziție SMC}{}
\begin{block}{Definiția informală a lui $\Swhile B \Sdo C$ (Kernighan \& Ritchie)}
Comanda $C$ se execută în mod repetat cât timp valoarea expresiei B ramâne $\Strue$.  Testul are loc înainte de execuția fiecărei comenzi.
\end{block}

\begin{block}{Reguli pentru $\Swhile$}
\onslide<2>
\[\begin{array}{lr@{{}\rightarrow{}}l}
\mbox{IterC}&
\c{(\Swhile B \Sdo C) \cdot c, r, s} & \c{B \cdot \Swhile \cdot c, B \cdot C\cdot r, s}
\\[1ex]
\mbox{IterT}&
\c{\Swhile \cdot c, \Strue \cdot  B \cdot C \cdot r, s} & \c{C \cdot (\Swhile B \Sdo C) \cdot c,  r, s}
\\[1ex]
\mbox{IterF}&
\c{\Swhile \cdot c, \Sfalse \cdot  B \cdot C \cdot r, s} & \c{c,  r, s}
\end{array}\]
\end{block}
\end{frame}
\begin{subsection}{Execuții folosind SMC}

\begin{frame}{Ce este o execuție?}
\begin{block}{Execuție}
E o secvență de tranziții de forma $ci \rightarrow^\ast cf$ unde $ci$ e o configurație inițială și $cf$ e o configurație finală
\end{block}
\begin{block}{Configurațiile inițiale}
Sunt de forma  $\c{C\cdot \Snil, \Snil, s}$ unde $C$ e o instrucțiune și $s$ o stare a memoriei.
\end{block}
\begin{block}{Configurațiile finale}
Sunt de forma  $\c{\Snil, \Snil, s}$ unde $s$ e o stare a memoriei.
\end{block}
\begin{block}{Configurațiile blocate}
Sunt acele configurații $c$ care nu sunt finale și care nu pot tranziționa, i. e.,  $c \not\rightarrow$  
\end{block}
\end{frame}

\begin{frame}[fragile]{O execuție}

\begin{minipage}{.35\columnwidth}
\begin{asciic}
 while (0 <= !x)  do
   (sum := !sum + !x ;
    x := !x - 1)
\end{asciic}
\end{minipage}
\hfill
\begin{minipage}{.63\columnwidth}
\begin{tabular}{ll}Fie& $C = {} \Swhile B \Sdo C'$,\\ unde &$B = 0 \terminal{<=} {\terminal{!} {\terminal{x}}}$,\\&$C' = C_1 \terminal{;} C_2$,\\ &$C_1 = {\terminal{sum}} \terminal{:=} {\terminal{!} \terminal{sum}} \terminal{+} {\terminal{!} \terminal{x}}$\\\hfill și & $C_2 =
{\terminal{x}} \terminal{:=} {\terminal{!} \terminal{x}} \terminal{-} 1 $
\end{tabular}
\end{minipage}
și fie $s = \{ x \mapsto 3, {\terminal{sum}} \mapsto 0\}$.  Atunci avem următoarea secvență de tranziții:
\begin{itemize}
\item[] 
$\begin{array}{l}\c{C\cdot \Snil, \Snil, s} \xrightarrow{Iter} \c{B\cdot \Swhile \cdot \Snil,  B\cdot C' \cdot \Snil, s} \xrightarrow{BoolOpC}
\\
 \c{0 \cdot (\terminal{!} x) \cdot \terminal{<=}\cdot \Swhile \cdot \Snil,  B\cdot C' \cdot \Snil, s} \xrightarrow{IntCt}
\\ 
\c{(\terminal{!} x) \cdot \terminal{<=}\cdot \Swhile \cdot \Snil,  0 \cdot B\cdot C' \cdot \Snil, s}  \xrightarrow{Loc}
\\  
\c{\terminal{<=}\cdot \Swhile \cdot \Snil,  3\cdot 0 \cdot B\cdot C' \cdot \Snil, s}
\xrightarrow{BoolOp}
\\  
\c{\Swhile \cdot \Snil,  \Strue \cdot B\cdot C' \cdot \Snil, s} 
 \xrightarrow{IterT}  \c{C' \cdot C \cdot \Snil,  \Snil, s} 
\end{array}$
\end{itemize}
\end{frame}

\begin{frame}[fragile]{O execuție (continuare)}{}
\only<beamer>{
\begin{minipage}{.35\columnwidth}
\begin{asciic}
 while (0 <= !x)  do
   (sum := !sum + !x ;
    x := !x - 1)
\end{asciic}
\end{minipage}
\hfill
\begin{minipage}{.63\columnwidth}
\begin{tabular}{ll}Fie& $C = {} \Swhile B \Sdo C'$,\\ unde &$B = 0 \terminal{<=} {\terminal{!} {\terminal{x}}}$,\\&$C' = C_1 \terminal{;} C_2$,\\ &$C_1 = {\terminal{sum}} \terminal{:=} {\terminal{!} \terminal{sum}} \terminal{+} {\terminal{!} \terminal{x}}$\\\hfill și & $C_2 =
{\terminal{x}} \terminal{:=} {\terminal{!} \terminal{x}} \terminal{-} 1 $
\end{tabular}
\end{minipage}
și fie $s = \{ x \mapsto 3, {\terminal{sum}} \mapsto 0\}$.  Atunci avem următoarea secvență de tranziții:
}
\begin{itemize}
\item[] $\begin{array}{l}
\c{C'\cdot C\cdot \Snil, \Snil, s} 
\xrightarrow{Secv} \c{C_1 \cdot C_2 \cdot C\cdot \Snil, \Snil, s}  
\xrightarrow{AtribC} 
\\
\c{({\terminal{!} \terminal{sum}} \terminal{+} {\terminal{!} \terminal{x}}) \cdot \terminal{:=} \cdot C_2 \cdot C\cdot \Snil, \terminal{sum}\cdot\Snil, s} 
\xrightarrow{IntOpC}
\\ 
\c{{\terminal{!} \terminal{sum}} \cdot {\terminal{!} \terminal{x}} \cdot \terminal{+} \cdot \terminal{:=} \cdot C_2 \cdot C\cdot \Snil, \terminal{sum}\cdot\Snil, s}
\xrightarrow{Loc}
\\
\c{{\terminal{!} \terminal{x}} \cdot \terminal{+} \cdot \terminal{:=} \cdot C_2 \cdot C\cdot \Snil, 0 \cdot \terminal{sum}\cdot\Snil, s} 
\xrightarrow{Loc}
\\
\c{\terminal{+} \cdot \terminal{:=} \cdot C_2 \cdot C\cdot \Snil, 3 \cdot 0 \cdot \terminal{sum}\cdot\Snil, s} 
\xrightarrow{IntOp}
\\
\c{\terminal{:=} \cdot C_2 \cdot C\cdot \Snil, 3 \cdot \terminal{sum}\cdot\Snil, s} 
\xrightarrow{IntOp} \c{C_2 \cdot C\cdot \Snil, \Snil, s'},
\end{array}$

unde  $s' = \{x\mapsto 3, s\mapsto 3\}$


\end{itemize}
\end{frame}


\begin{frame}[fragile]{O execuție (continuare)}{}
\only<beamer>{
\begin{minipage}{.35\columnwidth}
\begin{asciic}
 while (0 <= !x)  do
   (sum := !sum + !x ;
    x := !x - 1)
\end{asciic}
\end{minipage}
\hfill
\begin{minipage}{.63\columnwidth}
\begin{tabular}{ll}Fie& $C = {} \Swhile B \Sdo C'$,\\ unde &$B = 0 \terminal{<=} {\terminal{!} {\terminal{x}}}$,\\&$C' = C_1 \terminal{;} C_2$,\\ &$C_1 = {\terminal{sum}} \terminal{:=} {\terminal{!} \terminal{sum}} \terminal{+} {\terminal{!} \terminal{x}}$\\\hfill și & $C_2 =
{\terminal{x}} \terminal{:=} {\terminal{!} \terminal{x}} \terminal{-} 1 $
\end{tabular}
\end{minipage}
și fie $s' = \{ x \mapsto 3, {\terminal{sum}} \mapsto 3\}$.  Atunci avem următoarea secvență de tranziții:}
\begin{itemize}
\item[] $\begin{array}{l}
\c{C_2\cdot C\cdot \Snil, \Snil, s'}  
\xrightarrow{AtribC} \c{({\terminal{!} \terminal{x}} \terminal{-} 1) \cdot \terminal{:=} \cdot C\cdot \Snil, \terminal{x}\cdot\Snil, s'} 
\xrightarrow{IntOpC}
\\ 
\c{{\terminal{!} \terminal{x}} \cdot 1 \cdot \terminal{-} \cdot \terminal{:=} \cdot C\cdot \Snil, \terminal{x}\cdot\Snil, s'}
\xrightarrow{Loc} \c{1 \cdot \terminal{-} \cdot \terminal{:=} \cdot C\cdot \Snil, 3 \cdot \terminal{x}\cdot\Snil, s'} 
\xrightarrow{Int}
\\ 
\c{\terminal{-} \cdot \terminal{:=} \cdot C\cdot \Snil, 1 \cdot 3 \cdot \terminal{x}\cdot\Snil, s'} 
\xrightarrow{IntOp} \c{\terminal{:=} \cdot C\cdot \Snil, 2 \cdot \terminal{x}\cdot\Snil, s'} 
\xrightarrow{IntOp}
\\ 
\c{C\cdot \Snil, \Snil, \{x\mapsto 2, {\terminal{sum}} \mapsto 3\}} 
\rightarrow^\ast \c{C\cdot \Snil, \Snil, \{x\mapsto 1, {\terminal{sum}} \mapsto 5\}} 
\rightarrow^\ast
\\ 
\c{C\cdot \Snil, \Snil, \{x\mapsto 0, {\terminal{sum}} \mapsto 6\}} 
\rightarrow^\ast
\\  
\c{\Swhile \cdot \Snil,  \Sfalse \cdot B\cdot C' \cdot \Snil, \{x\mapsto 0, {\terminal{sum}} \mapsto 6\}} 
 \xrightarrow{IterF}
\\
\c{\Snil,  \Snil, \{x\mapsto 0, {\terminal{sum}} \mapsto 6\}} \not\rightarrow
\end{array}$
\end{itemize}
\end{frame}
\end{subsection}

\begin{frame}{Ce putem/vrem să demonstrăm despre mașina SMC}
\begin{itemize}
\vitem Mașina e deterministă?
\vitem Există execuții infinite?
\vitem Orice execuție pornind dintr-o stare inițială ajunge (dacă se termină) intr-o stare finală?
\begin{itemize}
\item Dacă nu, ce condiții adiționale asupra stărilor inițiale sunt necesare?
\end{itemize}
\end{itemize}
\end{frame}
\end{section}

\end{document}



