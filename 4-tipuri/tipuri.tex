\documentclass[xcolor=pdftex,romanian,colorlinks]{beamer}

\usepackage{../tslides}
\usepackage[all]{xy}

\title[SLP---Tipuri]{Tipuri}
\subtitle{Definirea unui sistem de tipuri} 

\begin{document}
\begin{frame}
  \titlepage
\end{frame}

\begin{frame}{Este IMP prea expresiv?}
\begin{itemize}
\item Sunt programe care n-aș vrea să le pot scrie, dar le pot?

\hfill $3 + \Sfalse$\hfill $\Sif 2 \Sthen {\Sskip} \Selse {\Sskip}$\hfill\;
\item Putem detecta programe greșite înainte de rulare?
\item Soluție: Sistemele de tipuri
\end{itemize}
\end{frame}

\begin{frame}{Sisteme de tipuri}{La ce folosesc?}

\begin{itemize}
\item Descriu programele „bine formate“
\item Pot preveni anumite erori
\begin{itemize}
\item folosirea variabilelor nedeclarate/neințializate
\item detectarea unor bucați de cod inaccesibile
\item erori de securitate
\end{itemize}
\item Ajută compilatorul 
\item Pot influența proiectarea limbajului
\end{itemize}
\vfill
\begin{block}{Scop (ideal)}
Progamele „bine formate“, i. e., cărora li se poate asocia un tip nu eșuează
\end{block}
\end{frame}

\begin{frame}{Sisteme de tipuri}{Intuiție}
\begin{itemize}
\item Vom defini o relație \structure{$\Gamma \vdash e : T$}
\item Citim \structure{$e$ are tipul $T$ dacă $\Gamma$}, unde
\item $\Gamma$ — tipuri asociate locațiilor din $e$
\end{itemize}

\begin{block}{Exemple}
\[\begin{array}{lll@{\;:\;}ll}
&\vdash& \Sif {\Strue} \Sthen 2 \Selse 3+4 & \Sint 
\\\\
l_1:\Sintref&\vdash&\Sif !l_1 \geq 3 \Sthen !l_1 \Selse 3& \Sint 
\\\\
&\not\vdash&3 + {\Sfalse}&T&\mbox{pentru orice } T
\\\\
&\not\vdash&\alert<2>{\Sif {\Strue} \Sthen 3 \Selse {\Sfalse}} & \Sint
\end{array}\]
\end{block}
\end{frame}

\begin{frame}{Tipuri în limbajul IMP}
\begin{block}{Tipurile expresiilor = tipurile valorilor}
\[T ::= {\Sint} \mid {\Sbool} \mid {\Sunit}\]
\end{block}
\begin{block}{Tipurile locațiilor}
\[T_{\it loc} ::= {\Sintref}\]
\end{block}
\vfill 
Fie $\mathbb{T}$ și $\mathbb{T}_{\it loc}$ mulțimile defiinite de $T$ și respectiv $T_{\it loc}$.
\end{frame}

\begin{frame}{$\Gamma$ --- Mediul de tipuri}
\begin{itemize}
\item Asociază tipuri locațiilor
\item Funcție parțială (finită) de la $\mathbb{L}$ la $\mathbb{T}_{\it loc}$, asemănătoare memoriei
\[\Gamma : \mathbb{L} \xrightarrow{\circ}\mathbb{T}_{\it loc}\]
\item \structure{Notație:} o listă de perechi locație-tip

\[l_1:\Sintref, \ldots, l_n:\Sintref\]
\end{itemize}
\vfill
\begin{block}{Observații pentru limbajul IMP}
\begin{itemize}
\item Toate locațiile din $\Gamma$ au același tip: $\Sintref$
\item Apariția unei locații în $\Gamma$ înseamnă că locația e de fapt definită
\end{itemize}
\end{block}
\end{frame}

\begin{frame}{IMP: Reguli pentru tipuri}{Expresii aritmmetice}
\begin{itemize}
\item[] $\reg[int]{\tjud{n}{\Sint}}{}{n \in \mathbb{Z}}$
\vitem[] $\reg[bool]{\tjud{b}{\Sbool}}{}{b\in \{\Strue,\Sfalse\}}$
\vitem[] $\reg[op+]{\tjud{e_1 + e_2}{\Sint}}{\tjud{e_1}{\Sint} \si \tjud{e_2}{\Sint}}{}$
\vitem[] $\reg[op$\leq$]{\tjud{e_1 \terminal{<=} e_2}{\Sbool}}{\tjud{e_1}{\Sint} \si \tjud{e_2}{\Sint}}{}$
\vitem[] $\reg[if]{\tjud{\Sif e_1 \Sthen e_2 \Selse e_3}{{T}}}{\tjud{e_1}{\Sbool} \si \tjud{e_2}{{T}} \si \tjud{e_3}{{T}}}{}$
\end{itemize}
\end{frame}

\begin{frame}{IMP: Reguli pentru tipuri}{Exemplu}
{Arătați că $\tjud[]{\Sif {\Sfalse} \Sthen 2 \Selse 3+4}{\Sint}$}
\onslide<2>

\[\reg[if]{\tjud[]{\Sif {\Sfalse} \Sthen 2 \Selse 3+4}{\Sint}}{\reg[bool]{\tjud[]{\Sfalse}{\Sbool}}{\done}{} \si \reg[int]{\tjud[]{2}{\Sint}}{\done}{} \si \tjud[]{3+4}{\Sint}}{}{}\]

unde
\[\reg[op+]{\tjud[]{3+4}{\Sint}}{\reg[int]{\tjud[]{3}{\Sint}}{\done}{}\si \reg[int]{\tjud[]{4}{\Sint}}{\done}{}}{}\]
\end{frame}

\begin{frame}{IMP: Reguli pentru tipuri}{Referințe}
\begin{itemize}
\item[] $\reg[atrib]{\tjud{l\terminal{:=}e}{\Sunit}}{\tjud{e}{\Sint}}{\Gamma(l) = \Sintref}$
\vitem[] $\reg[loc]{\tjud{!l}{int}}{}{\Gamma(l) = \Sintref}$

\vitem Pentru IMP, $\Gamma(l) = \Sintref$ poate fi citit ca: {\em locația $l$ e definită}
\end{itemize}
\end{frame}

\begin{frame}{IMP: Reguli pentru tipuri}{Programare imperativă}
\begin{itemize}
\item[] $\reg[skip]{\tjud{\Sskip}{\Sunit}}{}{}$
\vitem[] $\reg[secv]{\tjud{e_1\terminal{;} e_2}{T}}{\tjud{e_1}{\Sunit}\si \tjud{e_2}{T}}{}$
\vitem[] $\reg[while]{\tjud{\Swhile e_1 \Sdo e_2 \Sdone}{\Sunit}}{\tjud{e_1}{\Sbool} \si \tjud{e_2}{\Sunit}}{}$
\end{itemize}
\end{frame}

\begin{frame}{Proprietăți}{ale sistemului de tipuri pentru limbajul IMP}
\begin{theorem}[Proprietatea de a progresa] Dacă $\tjud{e}{T}$ și $\Dom(\Gamma) \subseteq \Dom(s)$ atunci $e$ este valoare sau $\c{e,s}$ poate progresa: există $e'$, $s'$ astfel încât $\Ss{\c{e,s}}{\c{e',s'}}$.
\end{theorem}
\vfill

\begin{theorem}[Proprietatea de conservare a tipului]
Dacă $\tjud{e}{T}$, $\Dom(\Gamma) \subseteq \Dom(s)$ și $\Ss{\c{e,s}}{\c{e',s'}}$, \\atunci 
$\tjud{e'}{T}$ și $\Dom(\Gamma) \subseteq \Dom(s')$.
\end{theorem}

\vfill
\begin{theorem}[Siguranță---programele bine formate nu se împotmolesc]
Dacă $\tjud{e}{T}$, $\Dom(\Gamma) \subseteq \Dom(s)$ și ${\c{e,s}}\longrightarrow^\ast{\c{e',s'}}$, atunci $e'$ este valoare sau există $e''$, $s''$ astfel încât $\Ss{\c{e',s'}}{\c{e'',s''}}$.
\end{theorem}

\end{frame}

\begin{frame}{Probleme computaționale}
\begin{block}{Verificarea tipului}
Date fiind $\Gamma$, $e$ și $T$, verificați dacă $\tjud{e}{T}$.
\end{block}

\begin{block}{Determinarea (inferarea) tipului}
Date fiind $\Gamma$ și $e$, găsiți (sau arătați ce nu există) un $T$ astfel încât $\tjud{e}{T}$.
\end{block}

\begin{itemize}
\item A doua problemă e mai grea în general decât prima
\item Algoritmi de inferare a tipurilor
\begin{itemize}
\item Colectează constrângeri asupra tipului
\item Folosesc metode de rezolvare a constrângerilor (programare logică)
\end{itemize}
\item Pentru limbajul nostru ambele probleme sunt ușoare
\end{itemize}
\end{frame}

\begin{frame}{Probleme computaționale}{Proprietăți}
\begin{theorem}[Determinarea tipului este decidabilă]
Date fiind $\Gamma$ și $e$, poate fi găsit (sau demonstrat că nu există) un $T$ astfel încât 
$\tjud{e}{T}$.
\end{theorem}
\vfill

\begin{theorem}[Verificarea tipului este decidabilă]
Date fiind $\Gamma$, $e$ și $T$, problema $\tjud{e}{T}$ este decidabilă.
\end{theorem}

\vfill
\begin{theorem}[Unicitatea tipului]
Dacă $\tjud{e}{T}$ și $\tjud{e}{T'}$, atunci $T=T'$.
\end{theorem}
\end{frame}

\begin{frame}{Argumente împotriva sistemelor de tipuri}
\begin{itemize}
  \item<2-> Sistemul de tipuri e prea restrictiv: programe „bune“ nu se compilează
      \[\Sif {\Sfalse} \Sthen 1 \Selse {\Strue}\]

  \vitem<3-> E pierdere de vreme să tot urmărești tipurile și să le modifici
     
     (limbaje de scripting)

  \vitem<4-> E prea mult de scris (e.g., tipuri STL in C++)
    
    Soluție:  Detectarea tipurilor

  \vitem<5-> Erorile de tipuri sunt greu/imposibil de citit

     Câteodată da --- a se vedea erorile STL \ldots

  \vitem<6> Tipurile nu mă lasă să scriu codul care îl vreau
\end{itemize}
\end{frame}

\end{document}



