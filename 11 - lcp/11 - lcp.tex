\documentclass[xcolor=pdftex,romanian,colorlinks]{beamer}

\usepackage{../tslides}
\usetikzlibrary{matrix}
\usepackage{wrapfig}

\title[PD---LCP]{LCP --- Limbajul Proceselor Comunicante}
\begin{document}
\maketitle


\begin{section}{LCP—Limbajul Proceselor Comunicante}

\begin{frame}{LCP}{Limbajul Proceselor Comunicante}
\begin{itemize}
\item Bazat pe CCS și $\pi$-calcul create de Robin Milner
\item Consideram un limbaj al expresiilor (intregi si booleene)
\item Adăugăm o nouă categorie sintactică, a constructorilor de proces 
\begin{itemize}
\item Paralelism  
\item Crearea canalelor de comunicare 
\item Trimitere de mesaje 
\item Primire de mesaje
\item Alegere (nedeterministă)
\end{itemize}
\end{itemize}
\end{frame}

\begin{frame}{Sintaxa}
\vspace{-5ex}\begin{syntaxBlock}{\Proc}
\alert{
\begin{itemize}
\item[]\renewcommand{\syntaxKeyword}{}
\syntax{\Proc\parallel\Proc \Smid \nu \Channel . \Proc \Smid  \Channel (\Variable).\Proc\Smid \overline{\Channel}\langle\AExp\rangle.\Proc\Smid \Proc + \Proc \Smid  \zero\Smid \structure{A}}{}
\syntaxCont{\Sif \BExp \Sthen \Proc \Selse \Proc}{}
\item[]\renewcommand{\defSort}{\AExp}
\syntax{\Variable \Smid \Int \Smid \AExp\iop \AExp}{}
\item[]\renewcommand{\defSort}{\BExp}
\syntax{\Bool \Smid \AExp\bop \AExp}{}
%\item[]\renewcommand{\defSort}{\Block}
%\syntax{\terminal{\{}\Stmt\terminal{\}} \Smid \terminal{\{}\terminal{\}}}{}

\item[]\renewcommand{\defSort}{\structure{A}}
\syntax{\mbox{constantă de proces  (macrocomandă)}}{} 
 
\item[]\renewcommand{\defSort}{\Variable}
\syntax{\mbox{variabilă dintr-o mulțime infinită fixată}}{} 
\\
\renewcommand{\defSort}{\Channel}
\syntax{\mbox{identificator de canal dintr-o mulțime infinită fixată}}{} 
\\
\renewcommand{\defSort}{\Int}
\syntax{\mbox{număr întreg}}{}
\\
\renewcommand{\defSort}{\Bool}
\syntax{\Strue \mid \Sfalse}{}
\\
\renewcommand{\defSort}{\iop}
\syntax{\terminal{+} \Smid \terminal{-}\Smid \ldots}{}
\renewcommand{\defSort}{\bop}
\syntax{\terminal{=} \Smid \terminal{<}\Smid \ldots}{}
%\syntaxCont{\terminal{and} \Smid \terminal{or}\Smid \ldots}{}
\end{itemize}
}
\end{syntaxBlock}
\end{frame}

\begin{frame}{Semantică intuitivă}
\begin{description}
\item[$p_1\parallel p_2$] Execută în paralel procesele $p_1$ și $p_2$
\vitem[$\nu c . p$] declară un canal nou $c$ cunoscut doar de $p$
\vitem[$c(x).p$] Citește în $x$ o valoare întreagă de pe canalul $c$, apoi continuă cu $p$
\vitem[$\out{c}{e}. p$] Trimite pe  canalul $c$ valoarea lui $e$, apoi continuă cu $p$
\vitem[$p_1 + p_2$] Alege unul dintre procesele $p_1$ și $p_2$ care poate continua
\vitem[$\zero$] Procesul vid / blocat / terminat
\vitem[$A$] simboluri (distincte) de macrocomenzi definite recursiv:
\[\left\{\begin{array}{lcl}
A_1 & \stackrel{\mbox{def}}{=} & P_1 \\
A_2 & \stackrel{\mbox{def}}{=} & P_2 \\
& \vdots & \\
\end{array}\right.\] 
$P_1,\ldots$ procese în care apar doar constantele $A_1,\ldots $
\end{description}
\end{frame}

\begin{subsection}{Example}
\begin{frame}
\only<beamer|handout>{\frametitle{Exemple}\framesubtitle{Funcții ca procese server}}
\only<article>{\frametitle{Funcții ca procese server}}
\begin{itemize}
\item $SQ_{i,o} \defeq i(x).\out{o}{x*x} \only<2->{. SQ_{i,o}}$
\item $\out{i}{5} . \out{i}{3} . \zero \parallel SQ_{i,stdout}$ sau, mai bine, $\nu i.(\out{i}{5} . \out{i}{3} . \zero \parallel SQ_{i,stdout})$ 

\vfill
\item<3-> $MUL_{i,o} \defeq \only<4->{i(x).i(y).\out{o}{ x*y} . MUL_{i,o}}$
\item<4-> $SQ'_{i,o} \defeq \only<5->{i(x).\nu i'.\nu o'.(\out{i'}{ x}.\out{i'}{ x}.o'(z).\out{o}{ z }.SQ'_{i,o} \parallel MUL_{i',o'} )}$

\vfill
\item<5-> $SUM_{i,o} \defeq \only<6>{i(x).\Sif x <= 0 \Sthen \out{o}{ 0 } . SUM_{i,o} \Selse \nu i'.\nu o'.(\out{i'}{ x-1 }.o'(s).\out{o}{ s + x } . SUM_{i,o} \parallel SUM_{i',o'})}$

\end{itemize}


\end{frame}

\begin{frame}{Nedeterminism}
\begin{itemize}
\item<2-> $RND_o \defeq \out{o}{0}.RND_o + \out{o}{1}.RND_o$

\vitem<3> $RND'_o \defeq \out{o}{0}.RND'_o + \nu c.(RND'_{c} \parallel c(x).\out{o}{x+1}.RND'_o)$
\end{itemize}


\end{frame}


\begin{frame}{Coadă de dimensiune finită/ Zonă de tampon}
\begin{itemize}
\item De dimensiune 1
\item<2-> $B1_{i,o} \defeq i(x).\out{o}{x}.B1_{i,o}$

\vitem<3-> De dimensiune 2
\item<4-> $B2_{i,o} \defeq \nu c. (B1_{i,c} \parallel B1_{c,o})$

\vitem<5-> De dimensiune 3
\item<6> $B3_{i,o} \defeq \nu c.\nu c'. (B1_{i,c} \parallel B1_{c,c'} \parallel B1_{c',o})$

\end{itemize}


\end{frame}

\end{subsection}
\begin{subsection}{Semantica formală}

\begin{frame}{Relația de tranziție}
Relația de tranziție semantică pentru LCP este de forma:
\[\c{p}\xrightarrow{\alpha}\c{p'}\]
unde
\begin{itemize}
\item $p$ și $p'$ sunt procese
\item $\alpha$ este o acțiune
\alert{
\renewcommand{\syntaxKeyword}{}
\syntax[\structure{\alpha}]{\epsilon \Smid \Channel(\Int) \Smid \out{\Channel}{\Int}}{}
}
unde $\Int$ e un număr întreg
\end{itemize}
\end{frame}

\begin{frame}{Paralelism și nedeterminism}
\begin{itemize}
\item[]$\reg[$\xrightarrow{\parallel_1}$]{\Ss[\alpha]{\c{p_1\parallel p_2}}{\c{p_1'\parallel p_2}}}{\Ss[\alpha]{\c{p_1}}{\c{p_1'}}}{}$ \hfill și \hfill $\reg[$\xrightarrow{\parallel_2}$]{\Ss[\alpha]{\c{p_1\parallel p_2}}{\c{p_1\parallel p_2'}}}{\Ss[\alpha]{\c{p_2}}{\c{p_2'}}}{}$ 

\vitem[]$\reg[$\xrightarrow{+_1}$]{\Ss[\alpha]{\c{p_1+ p_2}}{\c{p_1'}}}{\Ss[\alpha]{\c{p_1}}{\c{p_1'}}}{}$ \hfill și \hfill $\reg[$\xrightarrow{+_2}$]{\Ss[\alpha]{\c{p_1\parallel p_2}}{\c{p_2'}}}{\Ss[\alpha]{\c{p_2}}{\c{p_2'}}}{}$ 
\end{itemize}
\end{frame}

\begin{frame}{Comunicare}
$\reg[$\xrightarrow{in}$]{\Ss[c(n)]{\c{c(x).p}}{\c{p[n/x]}}}{}{}$ \hfill și \hfill $\reg[$\xrightarrow{out}$]{\Ss[\out{c}{n}]{\c{\out{c}{e}.p}}{\c{p}}}{}{\S{e}{n}}$ 

\vfill $\reg{\Ss[\epsilon]{\c{p_1\parallel p_2}}{\c{p_1' \parallel p_2'}}}{\Ss[c(n)]{\c{p_1}}{\c{p_1'}}\si \Ss[\out{c}{n}]{\c{p_2}}{\c{p_2'}}}{} \mbox{ și } \reg[$\xrightarrow{com}$]{\Ss[\epsilon]{\c{p_1\parallel p_2}}{\c{p_1' \parallel p_2'}}}{\Ss[\out{c}{n}]{\c{p_1}}{\c{p_1'}}\si \Ss[c(n)]{\c{p_2}}{\c{p_2'}}}{}$

\vfill
 $\reg[$\xrightarrow{\nu}$]{\Ss[\alpha]{\c{\nu c.p}}{\c{\nu c.p'}}}{\Ss[\alpha]{\c{p}}{\c{p'}}}{\alpha \not\in \{c(n),\out{c}{n}\mid n \mbox{ întreg}\}}$
\end{frame}

\begin{frame}{Constante (macrocomenzi) și condițional}
\begin{itemize}
\item[]$\reg[$\xrightarrow{con}$]{\Ss[\alpha]{\c{A}}{\c{p'}}}{\Ss[\alpha]{\c{p}}{\c{p'}}}{A \defeq p}$
\vitem[]$\reg[$\xrightarrow{ifT}$]{\Ss[\epsilon]{\c{\Sif b \Sthen p_1 \Selse p_2}}{\c{p_1}}}{}{\S{b}{\Strue}}$
\item[]$\reg[$\xrightarrow{ifF}$]{\Ss[\epsilon]{\c{\Sif b \Sthen p_1 \Selse p_2}}{\c{p_2}}}{}{\S{b}{\Sfalse}}$
\end{itemize}
\end{frame}

\begin{frame}{Vrem/putem să demonstrăm}
\begin{itemize}
\item $\c{SQ_{i,o}} \approx \c{SQ'_{i,o}}$, unde
\begin{itemize}
\item[] $SQ_{i,o} \defeq i(x).\out{o}{x*x} . SQ_{i,o}$
\item[] $SQ'_{i,o} \defeq {i(x).\nu i'.\nu o'.(\out{i'}{ x}.\out{i'}{ x}.o'(z).\out{o}{ z }.SQ'_{i,o} \parallel MUL_{i',o'} )}$
\item[] $MUL_{i,o} \defeq {i(x).i(y).\out{o}{ x*y} . MUL_{i,o}}$
\end{itemize}

\vitem $\c{B2} \not\approx \c{B2'}$, unde
\begin{itemize}
\item[] $B2 \defeq \nu c. (B1_{i,c} \parallel B1_{c,o})$
\item[] $B1_{i,o} \defeq i(x).\out{o}{x}.B1_{i,o}$
\item[] $B2' \defeq i(x).(\out{o}{x}.B2' + i(x').\out{o}{x}.\out{o}{x'}.B2')$

\end{itemize}
\end{itemize}
\end{frame}

\end{subsection}

\end{section}

\end{document}